%%%%%%%%%%%%%%%%%%%%%%%%%%%%%%%%%%%%%%%%%
% Beamer Presentation
% LaTeX Template
% Version 1.0 (10/11/12)
%
% This template has been downloaded from:
% http://www.LaTeXTemplates.com
%
% License:
% CC BY-NC-SA 3.0 (http://creativecommons.org/licenses/by-nc-sa/3.0/)
%
%%%%%%%%%%%%%%%%%%%%%%%%%%%%%%%%%%%%%%%%%

%----------------------------------------------------------------------------------------
%	PACKAGES AND THEMES
%----------------------------------------------------------------------------------------

\documentclass{beamer}


\mode<presentation> {

% The Beamer class comes with a number of default slide themes
% which change the colors and layouts of slides. Below this is a list
% of all the themes, uncomment each in turn to see what they look like.

%\usetheme{default}
%\usetheme{AnnArbor}
%\usetheme{Antibes}
%\usetheme{Bergen}
%\usetheme{Berkeley}
%\usetheme{Berlin}
%\usetheme{Boadilla}
%\usetheme{CambridgeUS}
%\usetheme{Copenhagen}
%\usetheme{Darmstadt}
%\usetheme{Dresden}
%\usetheme{Frankfurt}
%\usetheme{Goettingen}
%\usetheme{Hannover}
%\usetheme{Ilmenau}
%\usetheme{JuanLesPins}
%\usetheme{Luebeck}
\usetheme{Madrid}
%\usetheme{Malmoe}
%\usetheme{Marburg}
%\usetheme{Montpellier}
%\usetheme{PaloAlto}
%\usetheme{Pittsburgh}
%\usetheme{Rochester}
%\usetheme{Singapore}
%\usetheme{Szeged}
%\usetheme{Warsaw}

% As well as themes, the Beamer class has a number of color themes
% for any slide theme. Uncomment each of these in turn to see how it
% changes the colors of your current slide theme.

%\usecolortheme{albatross}
%\usecolortheme{beaver}
%\usecolortheme{beetle}
%\usecolortheme{crane}
%\usecolortheme{dolphin}
%\usecolortheme{dove}
%\usecolortheme{fly}
%\usecolortheme{lily}
%\usecolortheme{orchid}
%\usecolortheme{rose}
%\usecolortheme{seagull}
%\usecolortheme{seahorse}
%\usecolortheme{whale}
%\usecolortheme{wolverine}

%\setbeamertemplate{footline} % To remove the footer line in all slides uncomment this line
%\setbeamertemplate{footline}[page number] % To replace the footer line in all slides with a simple slide count uncomment this line

%\setbeamertemplate{navigation symbols}{} % To remove the navigation symbols from the bottom of all slides uncomment this line
}

\usepackage{macros}

%\usepackage{biblatex}
%\addbibresource{ijcai17.bib}
%\usepackage{natbib}
%\usepackage[round]{natbib}
%\usepackage[comma,numbers,sort&compress]{natbib}

%----------------------------------------------------------------------------------------
%	TITLE PAGE
%----------------------------------------------------------------------------------------

\title[Thresholding Bandits with Augmented UCB]{Thresholding Bandits with Augmented UCB} % The short title appears at the bottom of every slide, the full title is only on the title page

\author{Subhojyoti Mukherjee $^{1}$ \\ K.P. Naveen $^{2}$ \\ Nandan Sudarsanam $^{1,3}$ \\ Balaraman Ravindran $^{1,3}$ } % Your name
\institute[$^{1}$ IIT Madras \\ $^{2}$ IIT Tirupati] % Your institution as it will appear on the bottom of every slide, may be shorthand to save space
{
$^{1}$ Indian Institute of Technology Madras, \\ $^{2}$ Indian Institute of Technology Tirupati, \\ $^{3}$ Robert Bosch Centre for Data Science and AI \\ % Your institution for the title page
\medskip
%\textit{john@smith.com} % Your email address
}
%\date{\today} % Date, can be changed to a custom date
\date{August 24, 2017}

\begin{document}
\nocite{*}
\begin{frame}
\titlepage % Print the title page as the first slide
\end{frame}

\begin{frame}
\frametitle{Overview} % Table of contents slide, comment this block out to remove it
\tableofcontents % Throughout your presentation, if you choose to use \section{} and \subsection{} commands, these will automatically be printed on this slide as an overview of your presentation
\end{frame}

%----------------------------------------------------------------------------------------
%	PRESENTATION SLIDES
%----------------------------------------------------------------------------------------


%\section{Introduction}
%\begin{frame}
\frametitle{Introduction}
\begin{itemize}
\item<1-> The bandit problem is a sequential decision making process where at each timestep we have to choose one action or arm from a set of arms.
\item<2-> There is a specific reward distribution attached to each arm.  After pulling an arm we receive a reward from the reward distribution specific to the arm. 
\item<3-> After say pulling each arm once, we are presented with an \emph{exploration-exploitation}  trade-off, that is whether to continue to pull the arm for which we have observed the highest estimated reward till now(exploitation) or to explore a new arm(exploration). 
\item<4-> If we become too greedy and always exploit, we may miss the chance of actually finding the optimal arm and get stuck with a sub-optimal arm.
\end{itemize}
\end{frame}

\begin{frame}
\frametitle{Some practical applications}
\begin{itemize}
\item<1-> Selecting the best channel (out of several existing channels) for mobile communications in a very short duration.
\item<2-> Selecting a small set of best workers (out of a very large pool of workers) whose productivity is above a threshold.
\item<3-> Selecting the best possible route for a message to pass through in a peer-to-peer network connection.
\end{itemize}
\end{frame}

\begin{frame}
\frametitle{Why study bandits at all?}
\begin{itemize}
\item<1-> We know of $\epsilon$-greedy \cite{sutton1998reinforcement} algorithm, we can simply stick to it.
\item<2-> But $\epsilon$-greedy only gives us an asymptotic guarantee. There is no guarantee that in a highly regressive environment how $\epsilon$-greedy will behave. Can we be better in our search?
\item<3-> Bandits allows us to study this behavior in a more formal way giving us strict guarantees regarding the performance of our algorithm.
%\item<4-> They form the linking pieces of a larger problem.
\item<4-> They are easy to implement.    
\end{itemize}
\end{frame}



%\section{Stochastic Multi-Armed Bandit Problem}
%\begin{frame}
\frametitle{Stochastic Multi-Armed Bandit Problem}
\begin{itemize}
\item<1-> In stochastic multi-armed bandit problem, we are presented with a finite set of actions or arms. 
\item<2-> The rewards for each of the arms drawn from distributions are identical and independent random variables. 
\item<3-> The learner does not know the mean of the distributions, denoted by $r_{i}$. 
\item<4-> The learner has to find the optimal arm the mean of whose distribution is denoted by $r^{*}$ such that $r^{*}> r_{i}, \forall i\in A$.
\item<5-> The distributions for each of the arms are fixed throughout the time horizon. 
\end{itemize}
\end{frame}

\begin{frame}
\frametitle{Basic Notations}
\begin{itemize}
\item<1-> Goal: To minimize Regret
\item<2-> Average reward of best action is $r^{*}$ and any other action $i$ as $r_{i}$. There are $K$ total actions. $T_{i}(n)$ is number of times tried action $i$ is executed till $n$-timesteps.
\item<3-> Cumulative Regret: The loss we suffer because of not pulling the optimal arm till the total number of timesteps  $T$. 
\begin{align*}
R_{T}=r^{*}T - \sum_{i\in A} r_{i}T_{i}(T),
\end{align*}
\item<4-> The expected regret of an algorithm after $T$ rounds can be written as
\begin{align*}
\E[R_{T}]= \sum_{i=1}^K \E[T_{i}(T)] \Delta_i,
\end{align*}
\item<4-> $\Delta_{i}=r^{*}-r_{i}$ denotes the gap between the means of the optimal arm and of the $i$-th arm. 
\end{itemize}
\end{frame}

\begin{frame}
\frametitle{Another Notion of Regret}
\begin{itemize}
\item<1-> Goal: To minimize Regret
\item<2-> Can we have a policy which achieves the minimum regret among all the possible environments available?
\item<3-> This is called the worst case gap-independent regret or sometimes called the minimax regret.
\item<4-> It is generally found by setting all the gaps to equal values of order $O\left( 1/\sqrt{T} \right)$.
\item<5-> Also we will define the hardness parameter $H$ as $H=\sum_{i=1}^{K}\dfrac{1}{\Delta_{i}^2}$
\end{itemize}
\end{frame}

\section{Problem Definition}
\begin{frame}
\frametitle{Problem Definition}
\begin{itemize}
\item<1-> The primary aim in the thresholding bandit problem (TBP) is to identify the arms whose mean of the reward distribution is above a particular threshold $\tau$ given as input.
\item<2-> The above goal has to be achieved within $T$ timesteps of exploration and this is termed as a fixed-budget problem.
\item<3-> At the end of the given $T$ timesteps the learner must recommend a set of arms which (according to it) are the arms having reward mean above $\tau$.
\end{itemize}
\end{frame}

\begin{frame}
\frametitle{Problem Definition}
\begin{itemize}
\item<1-> We define the set $S_{\tau}=\lbrace i\in \mathcal{A}: r_{i}\geq \tau \rbrace$. 
%Note that, $S_\tau$ is the set of all arms whose reward mean is greater than $\tau$. Let 
\item<2-> $S_\tau^c$ denote the complement of $S_\tau$, i.e.,  $S_{\tau}^{c}=\lbrace i\in \mathcal{A}: r_{i} < \tau \rbrace$. 
\item<3-> Let $\hat{S}_{\tau}$ denote the recommendation of a learning algorithm after $T$ time units of exploration, while $\hat{S}_{\tau}^c$ denotes its complement.

%\item<4-> The performance of the learning agent is measured by the accuracy with which it can classify the arms into $S_{\tau}$ and $S_{\tau}^{c}$ after time horizon $T$. Equivalently, the \emph{loss} $\mathcal{L}(T)$ is defined as
%\begin{align*}
%\Ls (T) = \mathbb{I}\big(\lbrace S_{\tau}\cap \hat{S}_{\tau}^{c}\neq \emptyset\rbrace    \cup    \lbrace\hat{S}_{\tau}\cap S_{\tau}^{c}\neq \emptyset\rbrace\big).
%\end{align*}			

\item<4-> The goal of the learning agent is to minimize the expected loss:
\begin{align*}
\Ex[\Ls(T)] &= \Pb\big(\underbrace{\lbrace S_{\tau}\cap \hat{S}_{\tau}^{c} \neq \emptyset \rbrace}_{\textbf{Rejected good arms}}  \cup   \underbrace{\lbrace \hat{S}_{\tau}\cap S_{\tau}^{c} \neq \emptyset\rbrace}_{\textbf{Accepted bad arms}}\big) \\
%& = 1 - \Pb\big(\lbrace \hat{S}_{\tau}\cap {S}_{\tau}^{c} = \emptyset \rbrace \cap \lbrace \hat{S}_{\tau}^c \cap {S}_{\tau} = \emptyset \rbrace \big )
\end{align*}
\end{itemize}
\end{frame}

\begin{frame}
\frametitle{Challenges in the TBP Settings}
\begin{itemize}
\item<1-> The more number of arms' means are closer to the threshold the harder is to discriminate between them.
\item<2-> The lesser the budget is, the harder the problem becomes.
\item<3-> The higher the variance of the arms' the more difficult is to discriminate.
\end{itemize}
\end{frame}


\begin{frame}
\frametitle{Some practical applications}
\begin{itemize}
\item<1-> Selecting the best channels (out of several existing channels) for mobile communications in a very short duration whose qualities are above an acceptable threshold (see \cite{audibert2010best}).
\item<2-> Selecting a small set of best workers (out of a very large pool of workers) whose productivity is above a threshold.
\item<3-> In anomaly detection and classification (see {Locatelli et al. (2016)}).
\end{itemize}
\end{frame}

%The above TBP formulation has several applications, for instance, from areas ranging from anomaly detection and classification (see  \citet{locatelli2016optimal}) to industrial applications as well as in mobile communications (see \citet{audibert2010best})  where the users are to be allocated only those channels whose quality is above an acceptable threshold.

\section{Contribution}
\begin{frame}
\frametitle{Contributions}
\begin{itemize}
%\cite{DBLP:journals/corr/MukherjeeNSR17}
\item<1-> We propose the Augmented UCB (AugUCB) [{Mukherjee et al. (2017)}]  algorithm for the fixed-budget TBP setting.
\item<2-> AugUCB takes into account the empirical variances of the arms along with mean estimates.
\item<3-> It is the first variance-based arm elimination algorithm for the considered TBP settings. 
\item<4-> It addresses an open problem discussed in \cite{auer2010ucb} of designing an algorithm that can eliminate arms based on variance estimates.
\item<5-> We also define a new problem complexity which uses empirical variance estimates along with arm's mean for giving the theoretical bound.
\end{itemize}
\end{frame}


%Our theoretical contribution comprises proving an upper bound on the expected loss incurred by AugUCB (Theorem~\ref{Result:Theorem:1}).
%In Table \ref{tab:regret-bds} we compare the upper bound on the losses incurred by the various algorithms, including AugUCB. The terms $H_1, H_2$, $H_{CSAR,2}, H_{\sigma,1}$ and $H_{\sigma,2}$ represent various problem complexities, and are as defined in Section~\ref{results}. From Section~\ref{results} we note that, for all $K\ge8$, we have
%\begin{align*}
%\log\left(K\log K\right) H_{\sigma,2} > \log(2K) H_{\sigma,2} \ge H_{\sigma,1}.
%\end{align*}
%
%Thus, it follows that the upper bound for UCBEV is better than that for AugUCB. However, implementation of UCBEV algorithm requires $H_{\sigma,1}$ as input, whose computation is not realistic in practice. In contrast, our AugUCB algorithm requires no such complexity factor as input. Proceeding with the comparisons, we emphasize that the upper bound for  AugUCB is, in fact, not comparable with that of APT and CSAR; this is because the complexity term $H_{\sigma,2}$ is not explicitly comparable with either $H_1$ or $H_{CSAR,2}$. However, through extensive simulation experiments we find that AugUCB significantly outperforms both APT, CSAR and other non variance-based algorithms. AugUCB also outperforms UCBEV under explorations where non-optimal values of $H_{\sigma,1}$  are used. In particular, we consider experimental scenarios comprising large number of arms, with the variances of arms in $S_\tau$ being large. AugUCB, being variance based, exhibits superior performance under these settings.  
%%




%\section{Previous Works}
%

\begin{frame}
\frametitle{The Upper Confidence Bound (UCB) Approach}
\begin{itemize}
\item<1-> Since there is an initial uncertainty in the estimated mean ($\hat{r}_i$) introduce a confidence interval term $c_i$.
\item<2-> $c_i$ represents the uncertainty about $\hat{r}_i$. Higher the $c_i$, higher is the uncertainty.
\item<2-> $c_i$ ensures that the arm $i$ is properly explored and is gradually reduced with time as one pulls the arm $i$ more.
\item<3-> At every timestep pull arm $j\in \argmax_{i\in A} \lbrace \hat{r}_i + c_i\rbrace$ and this will ensure that proper exploration is done. 
\end{itemize}
\end{frame}

\begin{frame}
\frametitle{The UCB Approach}
\begin{figure}
%\caption{UCB Intuition}
\includegraphics[scale=0.3]{img/UCB_Drawing.png}
\end{figure}
\end{frame}


%\begin{frame}
%\frametitle{Previous Works (Pure Exploration)}
%\begin{itemize}
%\item<1-> The TBP problem also falls within the larger area called the Pure Exploration problem.
%\item<2-> In pure exploration problems the learner has to output a set of recommendations either with high confidence (fixed confidence) or after a specified number of rounds (fixed budget).
%\item<3-> Our considered TBP is a fixed budget pure exploration problem.
%\item<4-> Both APT and AugUCB reuses several ideas from Pure exploration problem. 
%\end{itemize}
%\end{frame}
%
%\begin{frame}
%\frametitle{Previous Works (Diagram)}
%\centering
%\begin{figure}
%\includegraphics[scale=0.3]{img/Settings2.png}
%\caption{TBP place within SMAB and Pure exploration}
%\end{figure}
%\end{frame}

%\begin{frame}
%\frametitle{Approach of UCB-Improved (UCB-Imp)}
%\begin{itemize}
%\item<1-> There is a strong relation between UCB-Imp \cite{auer2010ucb} and AugUCB where the former is used to find \emph{a single optimal arm as quickly as possible}.
%\item<2-> The basic idea of UCB-Improved is to divide the horizon into phases or rounds and initialize parameters.
%\item<3-> Pull all surviving arms equal number of times during a round.
%\item<4-> At the end of the round eliminate some sub-optimal arms (as judged by learner) based on elimination criteria.
%\item<5-> Reset parameters and proceed to next round.
%\end{itemize}
%\end{frame}
%
%\begin{frame}
%\frametitle{UCB-Improved (\cite{auer2010ucb})}
%\begin{algorithm}[H]
%\caption{UCB-Improved}
%\small
%\begin{algorithmic}[1]
%\State {\bf Input:} Time horizon $T$
%\State {\bf Initialization:} Set $B_{0}:=A$ and ${\epsilon}_{0}:=1$.
%\For{$m=0,1,..\big \lfloor \dfrac{1}{2}\log_{2} \dfrac{T}{e}\big\rfloor$}	
%\State Pull each arm in $B_m$, $n_{m}=\bigg\lceil\dfrac{2\log{( T{\epsilon}_{m}^{2})}}{{\epsilon}_{m}}\bigg\rceil$ number of times.
%%so that the total  it has been pulled is
%\ArmElim
%\State For each $i \in B_{m}$, delete arm ${i}$ from $B_{m}$ if,
%\begin{align*}
%\hat{r}_{i} + \sqrt{\dfrac{\log{(T{\epsilon}_{m}^{2})}}{2 n_{m}}}  < \max_{{j}\in B_{m}}\bigg\lbrace\hat{r}_{j} -\sqrt{\dfrac{\log{( T{\epsilon}_{m}^{2})}}{2 n_{m}}} \bigg\rbrace
%\end{align*}
%\EndArmElim
%%\ResParam
%\State Set ${\epsilon}_{m+1}:=\dfrac{{\epsilon}_{m}}{2}$, Set $B_{m+1}:=B_{m}$
%%\EndResParam
%\State Stop if $|B_{m}|=1$ and pull ${i}\in B_{m}$ till $T$ is reached.
%\EndFor
%\end{algorithmic}
%\end{algorithm}
%\end{frame}

\begin{frame}
\frametitle{Approach of UCB-Improved (UCB-Imp)}
\begin{figure}
%\caption{UCB Imp Approach}
\includegraphics[scale=0.25]{img/Ucb-Imp.png}
\end{figure}
\end{frame}



\begin{frame}
\frametitle{Intuition of UCB-Improved (UCB-Imp)}
\begin{figure}
%\caption{UCB Imp Intuition}
\includegraphics[scale=0.3]{img/Ucb_Imp_intuition.png}
\end{figure}
\end{frame}


\begin{frame}
\frametitle{APT Approach}
\begin{itemize}
\item<1-> The Anytime Parameter Free (APT) [{Locatelli et al. (2016)}] algorithm \textit{was proposed for TBP setting} in ICML 2016. 
\item<2-> This algorithm uses only mean estimation to find the $S_{\tau}$. 
\item<3-> Theoretically they proved this algorithm to be almost optimal when only mean estimation is used as a metric of comparison.
\item<4-> Empirically it outperformed other state-of-the-art algorithms which were modified to perform in the TBP setting.  
\end{itemize}
\end{frame}

\begin{frame}
\frametitle{APT Algorithm}
\begin{algorithm}[H]
\caption{APT}
\begin{algorithmic}
\State {\bf Input:} Time horizon $T$, threshold $\tau$, tolerance factor $\epsilon\geq 0$
\State Pull each arm once
\vspace{-3mm}
\State \For{$t=K+1,..,T$}
\State Pull arm $j\in\argmin_{i\in A}\Big\lbrace \left(|\hat{r}_{i} - \tau | + \epsilon\right)\sqrt{n_i}\Big\rbrace$ and observe the reward for arm $j$.
\EndFor
\State \textbf{Output:} $\hat{S}_{\tau}=\lbrace i: \hat{r}_{i}\geq \tau \rbrace$.
\end{algorithmic}
\end{algorithm}
\end{frame}


\begin{frame}
\frametitle{Intuition of APT}
\begin{figure}
%\caption{APT Intuition}
\includegraphics[scale=0.278]{img/APT_intuition.png}
\end{figure}
\end{frame}





%\begin{frame}
%\frametitle{Some technical details of UCB-Improved}
%\begin{itemize}
%\item<1-> We do not know the true means $r_i ,\forall i\in A$ of the distributions so we estimate it by the ${\epsilon}$ by initializing it from $1$.
%\item<2-> All rewards are assume to be bounded between $[0,1]$ and so $\Delta_{i} = (r^* - r_i)\in [0,1],\forall i\in A$ as well.
%\item<3-> UCB-Improved has fixed confidence interval  $c_{m}=\sqrt{\dfrac{\log{(T{\epsilon}_{m}^{2})}}{2 n_{m}}}$ for all arms in a particular round.
%\item<4-> $c_m$ ensures that whenever ${\epsilon}_{m}<\dfrac{\Delta_i}{2}$ in the $m$-th round, the arm $i$ gets eliminated.
%\end{itemize}
%\end{frame}

\section{AugUCB}
\begin{frame}
\frametitle{AugUCB algorithm (Intuition, Arm pulling)}
\begin{itemize}
\item We define $\Delta_i = |r_i - \tau| $ . 
\item It is risky to eliminate arm $i$ while $\hat{r}_i$ is inside \emph{Margin}. 
\item Confidence interval $s_i$ will make sure arm $i$ is not eliminated while inside Margin with a high probability. 
\end{itemize}

\begin{figure}
\caption{AugUCB Intuition (Arm pulling)}
\includegraphics[scale=0.178]{img/SeminarThresholdBandit.png}
\end{figure}
\end{frame}

\begin{frame}
\frametitle{AugUCB algorithm (Intuition, Arm Elimination)}
\begin{itemize}
\item It is risky to eliminate arm $i$ while $\hat{r}_i$ is inside \emph{Margin}. 
\item Confidence interval $s_i$ will make sure arm $i$ is not eliminated while inside Margin with a high probability. 
\end{itemize}

\begin{figure}
\caption{AugUCB Intuition (Arm Elimination)}
\includegraphics[scale=0.178]{img/ArmElim1.png}
\end{figure}
\end{frame}

\begin{frame}
\frametitle{AugUCB algorithm (Intuition, Arm Elimination)}
\begin{figure}
\caption{AugUCB Intuition (Arm Elimination)}
\includegraphics[scale=0.178]{img/ArmElim2.png}
\end{figure}
\end{frame}


\begin{frame}
\frametitle{AugUCB algorithm}
\begin{itemize}
\item<1-> Like UCB-Imp, AugUCB also divides the time budget $T$ into rounds.
\item<2-> A crucial difference is that in every round instead of pulling all the arms equal number of times we pull the arm that minimizes $j\in\argmin_{i\in B_{m}}\Big\lbrace |\hat{r}_{i} - \tau | - 2s_{i}\Big\rbrace$ (like APT). 
\item<3-> At every timestep now we run the arm elimination check to eliminate sub-optimal arms.
\item<4-> At the end of the phase we reset the parameters. 
\item<5-> Note that the length of the phase, the exploration parameters and the confidence interval term $s_i  = \sqrt{\frac{\rho\psi_m (\hat{v}_{i}+1) \log ( T \epsilon_{m})}{4 n_{i}}}$ are set through detailed theoretical analysis. 
\end{itemize}
\end{frame}

\begin{frame}[allowframebreaks]
\frametitle{AugUCB algorithm}
%\begin{algorithm}[H]
%\caption{AugUCB}
%\label{alg:augucb}
\begin{algorithmic}
\State {\bf Input:} Time budget $T$; parameter $\rho$; threshold $\tau$
\State {\bf Initialization:} $B_{0}=\mathcal{A}$; $m=0$; $\epsilon_{0}=1$;
\begin{small}
\begin{align*}
M&=\left\lfloor \frac{1}{2}\log_{2} \frac{T}{e}\right\rfloor; 
\hspace{2mm}\psi_{0}=\frac{T\epsilon_{0}}{128\Big(\log(\frac{3}{16}K\log K)\Big)^2}; \\
\ell_{0}&=\left\lceil \frac{2\psi_0\log( T\epsilon_{0})}{\epsilon_{0}} \right\rceil ;
\hspace{2mm}N_{0}=K\ell_{0}
\end{align*}
\end{small}
\State Pull each arm once
\vspace{-3mm}
\State \For{$t=K+1,..,T$}
\State Pull arm $j\in\argmin_{i\in B_{m}}\Big\lbrace |\hat{r}_{i} - \tau | - 2s_{i}\Big\rbrace$
\vspace{-4.5mm}
\State \For{$i\in B_m$}
\vspace{-4.5mm}
\State \If{$(\hat{r}_{i} + s_i  < \tau - s_i)$ or $(\hat{r}_{i} - s_i > \tau + s_i)$}
\State $B_m\leftarrow B_m\backslash\{i\}$\hspace{4mm} (Arm deletion)
\EndIf
\EndFor
\vspace{-2mm}
\State \If{$t\geq N_{m}$ and $m \leq M$}
%\ResetParam
\State \textbf{Reset Parameters}
\State $\epsilon_{m+1}\leftarrow\frac{\epsilon_{m}}{2}$
\State $B_{m+1} \leftarrow B_{m}$
\State $\psi_{m+1}\leftarrow \frac{T\epsilon_{m+1}}{128(\log(\frac{3}{16}K\log K))^{2}}$
\State $\ell_{m+1}\leftarrow\left\lceil \frac{2\psi_{m+1}\log( T\epsilon_{m+1})}{\epsilon_{m+1}} \right\rceil$
\State $N_{m+1} \leftarrow t + |B_{m+1}|\ell_{m+1}$
\State $m \leftarrow m+1$
\EndIf
\EndFor
\State \textbf{Output:} $\hat{S}_{\tau}=\lbrace i: \hat{r}_{i}\geq \tau \rbrace$.
\end{algorithmic}
%\end{algorithm}
\end{frame}

\section{Theoretical Analysis}
%\begin{frame}
%\frametitle{Problem Complexity}
%\begin{itemize}
%%\item<1-> We must delve into the notion of hardness which come from the general pure exploration bandit literature.
%\item<1-> $\Delta_i= |r_i - \tau |$ as in Locatelli et al. (2016).
%\item<2-> $H_{1} = \sum_{i=1}^{K}\dfrac{1}{\Delta_{i}^{2}}$ and $
%H_{2} =\min_{i\in \mathcal{A}}\dfrac{i}{{\Delta_{(i)}^{2}}} $ where $\Delta_{(i)}$ is an increasing ordering of ${\Delta}_{i}$.
%\item<3-> From {Audibert and Bubeck (2010)} the relationship between $H_1$ and $H_2$ can be derived as,
%\begin{align*}
%H_{2} \leq H_{1}\leq \log(2K)H_{2} 
%%\mbox{ and }
% %H_1 \leq \log(K)H_{CSAR,2}.
%\end{align*}
%\end{itemize}
%\end{frame}
%
%\begin{frame}
%\frametitle{Problem Complexity}
%\begin{itemize}
%\item<1-> For a variance aware algorithm, there is $H_{\sigma , 1}$ (as in {Gabillon et al. (2011)}) that incorporates reward variances into its expression as:
%\begin{align*}
% H_{\sigma,1}=\sum_{i=1}^{K}\frac{\sigma_{i}+\sqrt{\sigma_{i}^{2}+(16/3)\Delta_{i}}}{\Delta_{i}^{2}}.
%\end{align*}
%
%\item<2-> Finally, analogous to $H_{2}$, we introduce $H_{\sigma,2}$, such that $
%H_{\sigma,2}=\max_{i\in \mathcal{A}} \frac{i}{\tilde{\Delta}_{(i)}^{2}}$ , where $\tilde{\Delta}_{i}^{2}=\frac{\Delta_{i}^{2}}{\sigma_{i}+\sqrt{\sigma_{i}^{2}+(16/3)\Delta_{i}}}$,  $(\tilde{\Delta}_{(i)})$ is an increasing ordering of $(\tilde{\Delta}_{i})$.
%
%\item<3-> From {Audibert and Bubeck (2010)}, we can show that
%\begin{align*}
%H_{\sigma,2}\le H_{\sigma,1} \le \log(2K) H_{\sigma,2}.
%\end{align*}
%
%
%\item<4-> Note that $H_1 , H_2 $ and $H_{\sigma,1}, H_{\sigma,2}$ are not directly comparable to each other except in a special case when variances and gaps $(\Delta_i)$ are very low we can say that $H_{\sigma,1} < H_{1} $.
%
%\end{itemize}
%\end{frame}



\begin{frame}
\frametitle{Expected Loss of AugUCB}

\begin{theorem}
For $K\geq 4$ and $\rho={1}/{3}$,
the expected loss of the AugUCB algorithm is given by,
\begin{align*}
\E[\Ls(T)]
& \leq 2KT \exp\bigg(- \frac{T}{4096 \log( K\log K) H_{\sigma,2}} \bigg).
\end{align*}
\end{theorem}


\begin{table}[b]
\caption{AugUCB vs.\ State of the art}
\label{tab:regret-bds}
\begin{center}
\begin{tabular}{|p{1.5cm}|p{6.4cm}|p{1.5cm}|}
% \toprule
\hline
Algorithm  & Upper Bound on Expected Loss & Oracle \\
% \midrule
\hline
%\hline
AugUCB      &$ \exp\left(- \frac{T}{4096 \log(K\log K)H_{\sigma,2}} + \log\left(2KT\right) \right) $ & No\\
%\hline
\hline
UCBEV		&$\exp\left(-\frac{1}{512}\frac{T-2K}{H_{\sigma,1}} + \log\left(6KT\right)\right)$ & Yes\\
%\midrule
%\hline
\hline
APT         &$\exp\left(-\frac{T}{64 H_1}+2\log((\log(T)+1)K)\right)$ & No\\
% \midrule
%\hline
\hline
UCBE		&$\exp\left(-\frac{T-K}{18 H_1} - 2\log(\log(T)K\right)$ &  Yes\\
%\midrule
\hline

%\bottomrule
\end{tabular}
\end{center}
\end{table}
\end{frame}

%\begin{frame}
%\frametitle{Concentration Bounds}
%\begin{itemize}
%\item<1-> Let $X_{1}, . . . , X_{n}$ be random variables with common support $[0, 1]$ and such that $E[X_{t} |X_{1}, . . . , X_{t-1}] = r_i$. Let $\hat{r}_i = \dfrac{X_{1} +,....,+ X_{n}}{n}$. Then for all $c \geq 0$,
%\begin{align*}
%\mathbb{P} \lbrace \hat{r}_i \geq r_i + c \rbrace \leq \exp{(-2c^{2}n)}\\
%\mathbb{P}\lbrace \hat{r}_i \leq r_i - c \rbrace \leq \exp{(-2c^{2}n)}
%\end{align*}
%
%\item<2-> Along with the above information if we know that $\text{Var}[X_{t} |X_{1}, . . . , X_{t-1}]=\sigma_i^2$ then Bernstein inequality gives us,
%\begin{align*}
%\mathbb{P} \lbrace \hat{r}_i \geq r_i + c \rbrace \leq \exp{(-\dfrac{c^{2}n}{2\sigma_i^2 +\frac{2c}{3}})}\\
%\mathbb{P}\lbrace \hat{r}_i \leq r_i - c \rbrace \leq \exp{(-\dfrac{c^{2}n}{2\sigma_i^2 +\frac{2c}{3}})}
%\end{align*}
%
%\end{itemize}
%\end{frame}

%\begin{frame}
%\frametitle{Sketch of the proof}
%\begin{figure}
%%\caption{AugUCB arm elimination}
%\includegraphics[scale=0.278]{img/ArmElim3.png}
%\end{figure}
%\end{frame}





\section{Experiments}
\begin{frame}
\frametitle{Finally, experiment!!!}
\begin{figure}[!tbp]
    \centering
    \begin{tabular}{cc}
    \setlength{\tabcolsep}{0.1pt}
    \subfigure[0.25\textwidth][Experiment $1$: $20$ Bernoulli-distributed arms with $r_{i_{{i}\neq {*}}}=0.07$ and $r^{*}=0.1$.]
    {
    		\pgfplotsset{
		tick label style={font=\Large},
		label style={font=\Large},
		legend style={font=\Large},
		ylabel style={yshift=32pt},
		%legend style={legendshift=32pt},
		}
        \begin{tikzpicture}[scale=0.45]
      	\begin{axis}[
		xlabel={timestep},
		ylabel={Cumulative Regret},
		grid=major,
        %clip mode=individual,grid,grid style={gray!30},
        clip=true,
        %clip mode=individual,grid,grid style={gray!30},
  		legend style={at={(0.5,1.5)},anchor=north, legend columns=3} ]
      	% UCB
		\addplot table{results/NewExpt/Expt1/UCBV01_comp_subsampled.txt};
		\addplot table{results/NewExpt/Expt1/UCB01_comp_subsampled.txt};
		\addplot table{results/NewExpt/Expt1/KLUCB01_comp_subsampled.txt};
		\addplot table{results/NewExpt/Expt1/MOSS01_comp_subsampled.txt};
		\addplot table{results/NewExpt/Expt1/DMED01_comp_subsampled.txt};
		\addplot table{results/NewExpt/Expt1/EclUCB01_1_comp_subsampled.txt};
		\addplot table{results/NewExpt/Expt1/TS01_comp_subsampled.txt};
		\addplot table{results/NewExpt/Expt1/OCUCB01_comp_subsampled.txt};
		%\addplot table{results/NewExpt/Expt1/EclUCB011_comp_subsampled.txt};
      	\legend{UCB-V,UCB1,KL-UCB,MOSS,DMED,EClusUCB,TS,OCUCB}      	
      	\end{axis}
      	\end{tikzpicture}
  		\label{fig:1}
    }
    &
    \subfigure[0.25\textwidth][Experiment $2$: $100$ Gaussian-distributed arms with $r_{i_{{i}\neq {*}:1-33}}=0.1$, $r_{i_{{i}\neq {*}:34-99}}=0.6$ and $r^{*}_{i=100}=0.9$. ]
    {
    		\pgfplotsset{
		tick label style={font=\Large},
		label style={font=\Large},
		legend style={font=\Large},
		}
        \begin{tikzpicture}[scale=0.45]
        \begin{axis}[
		xlabel={timestep},
		ylabel={Cumulative Regret},
        %clip mode=individual,grid,grid style={gray!30},
       	grid=major,
       	clip=true,
  		legend style={at={(0.5,1.5)},anchor=north, legend columns=3} ]
      	% UCB
        \addplot table{results/NewExpt/Expt2_2/UCB01_comp_subsampled.txt};
		\addplot table{results/NewExpt/Expt2_2/clUCB01_comp_subsampled.txt};
		\addplot table{results/NewExpt/Expt2_2/MedElim_comp_subsampled.txt};
		\addplot table{results/NewExpt/Expt2_2/MOSS01_comp_subsampled.txt};
		\addplot table{results/NewExpt/Expt2_2/EclUCB01_comp_subsampled.txt};
		\addplot table{results/NewExpt/Expt2_2/OCUCB01_comp_subsampled.txt};
		\addplot table{results/NewExpt/Expt2_2/UCBR01_comp_subsampled.txt};
		%\addplot table{results/NewExpt/Expt2_2/EclUCB01_p_1_comp_subsampled.txt};
		\legend{UCB1,ClusUCB,Med-Elim,MOSS,EClusUCB,OCUCB,UCB-Imp}
      	%\legend{UCB1,ClusUCB,Med-Elim,MOSS,EClusUCB,OCUCB,UCB-Imp,EClusUCB-AE}
      	\end{axis}
      	\end{tikzpicture}
   		\label{fig:2}
    }
    \end{tabular}
    \caption{Cumulative regret for various bandit algorithms on two stochastic K-armed bandit environments. }
    \label{fig:karmed}
\end{figure}
\end{frame}



\section{Conclusion}
\begin{frame}
\frametitle{Conclusion ({Chapter 6})}
\begin{itemize}
\item<1-> We proposed the EUCBV algorithm for the SMAB setting which uses variance and mean estimation along with arm elimination to give an order-optimal theoretical guarantee.
\item<2-> We proposed the AugUCB algorithm for the fixed budget TBP  which uses variance estimation and arm elimination to give improved theoretical and experimental guarantees than mean estimation based algorithms.
\item<3-> Further studies are required to establish a lower bound on the expected loss of AugUCB.
\item<4-> A more detailed analysis of the non-uniform arm selection and parameter selection is also required for both AugUCB and EUCBV.
\end{itemize}
\end{frame}

\begin{frame}
\frametitle{Papers based on Thesis}
\begin{itemize}
\item Subhojyoti Mukherjee, K.P.~Naveen, Nandan Sudarsanam, and Balaraman Ravindran, "\textit{Thresholding Bandit with Augmented UCB}", {\em Proceedings of the Twenty-Sixth International Joint Conference on
               Artificial Intelligence, {IJCAI} 2017, Melbourne, Australia, August
               19-25, 2017,2515-2521}.
\item Subhojyoti Mukherjee, K.P.~Naveen, Nandan Sudarsanam, and Balaraman Ravindran, "\textit{Efficient UCBV: An Almost Optimal Algorithm using Variance Estimates}", {\em To appear in Proceedings of the Thirty-Second Association for the Advancement of Artificial Intelligence, {AAAI} 2018, New Orleans, Louisiana, USA, February 2-7}.
\end{itemize}
\end{frame}

%\section{References}
%\begin{frame}[allowframebreaks]
%\frametitle{References}
%%\bibliographystyle{named} 
%\bibliographystyle{plainnat} 
%%\bibliographystyle{dinat}
%\bibliography{ijcai17}
%\end{frame}


%------------------------------------------------

\begin{frame}
\Huge{\centerline{Thank You}}
\end{frame}

%----------------------------------------------------------------------------------------

\end{document} 