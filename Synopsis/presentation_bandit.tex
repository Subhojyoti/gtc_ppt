%%%%%%%%%%%%%%%%%%%%%%%%%%%%%%%%%%%%%%%%%
% Beamer Presentation
% LaTeX Template
% Version 1.0 (10/11/12)
%
% This template has been downloaded from:
% http://www.LaTeXTemplates.com
%
% License:
% CC BY-NC-SA 3.0 (http://creativecommons.org/licenses/by-nc-sa/3.0/)
%
%%%%%%%%%%%%%%%%%%%%%%%%%%%%%%%%%%%%%%%%%

%----------------------------------------------------------------------------------------
%	PACKAGES AND THEMES
%----------------------------------------------------------------------------------------

\documentclass{beamer}


\mode<presentation> {

% The Beamer class comes with a number of default slide themes
% which change the colors and layouts of slides. Below this is a list
% of all the themes, uncomment each in turn to see what they look like.

%\usetheme{default}
%\usetheme{AnnArbor}
%\usetheme{Antibes}
%\usetheme{Bergen}
%\usetheme{Berkeley}
%\usetheme{Berlin}
%\usetheme{Boadilla}
%\usetheme{CambridgeUS}
%\usetheme{Copenhagen}
%\usetheme{Darmstadt}
%\usetheme{Dresden}
%\usetheme{Frankfurt}
%\usetheme{Goettingen}
%\usetheme{Hannover}
%\usetheme{Ilmenau}
%\usetheme{JuanLesPins}
%\usetheme{Luebeck}
\usetheme{Madrid}
%\usetheme{Malmoe}
%\usetheme{Marburg}
%\usetheme{Montpellier}
%\usetheme{PaloAlto}
%\usetheme{Pittsburgh}
%\usetheme{Rochester}
%\usetheme{Singapore}
%\usetheme{Szeged}
%\usetheme{Warsaw}

% As well as themes, the Beamer class has a number of color themes
% for any slide theme. Uncomment each of these in turn to see how it
% changes the colors of your current slide theme.

%\usecolortheme{albatross}
%\usecolortheme{beaver}
%\usecolortheme{beetle}
%\usecolortheme{crane}
%\usecolortheme{dolphin}
%\usecolortheme{dove}
%\usecolortheme{fly}
%\usecolortheme{lily}
%\usecolortheme{orchid}
%\usecolortheme{rose}
%\usecolortheme{seagull}
%\usecolortheme{seahorse}
%\usecolortheme{whale}
%\usecolortheme{wolverine}

%\setbeamertemplate{footline} % To remove the footer line in all slides uncomment this line
%\setbeamertemplate{footline}[page number] % To replace the footer line in all slides with a simple slide count uncomment this line

%\setbeamertemplate{navigation symbols}{} % To remove the navigation symbols from the bottom of all slides uncomment this line
}

\usepackage{macros}
%\usepackage{enumitem}
%\usepackage{biblatex}
%\addbibresource{ijcai17.bib}
%\usepackage{natbib}
%\usepackage[round]{natbib}
%\usepackage[comma,numbers,sort&compress]{natbib}

%----------------------------------------------------------------------------------------
%	TITLE PAGE
%----------------------------------------------------------------------------------------

\title[Finite-time Analysis of Frequentist Strategies for Multi-armed Bandits]{Finite-time Analysis of Frequentist Strategies for Multi-armed Bandits} % The short title appears at the bottom of every slide, the full title is only on the title page

\author{Subhojyoti Mukherjee \\ CS15S300 \\ Guide: Dr. Balaraman Ravindran \\ Co-Guide: Dr. Nandan Sudarsanam \\  Collaborator: Dr. K.P. Naveen} % Your name
\institute[IIT Madras] % Your institution as it will appear on the bottom of every slide, may be shorthand to save space
{
IIT Madras \\ % Your institution for the title page
\medskip
%\textit{john@smith.com} % Your email address
}
\date{\today} % Date, can be changed to a custom date

\begin{document}
\nocite{*}
\begin{frame}
\titlepage % Print the title page as the first slide
\end{frame}

\begin{frame}
\frametitle{Overview} % Table of contents slide, comment this block out to remove it
%\tableofcontents % Throughout your presentation, if you choose to use \section{} and \subsection{} commands, these will automatically be printed on this slide as an overview of your presentation
\centering
    \begin{tabular}{|p{5cm}|p{5cm}|}
    \hline
SMAB Setting (Part 1) & TBP Setting (Part 2) \\
% \midrule
\hline
    \begin{itemize}
    \item Problem Definition
    \item Contributions
    \item EUCBV algorithm
    \item Theory
    \item Experiments
    \end{itemize}
    &
    \begin{itemize}
    \item Problem Definition
    \item Contributions
    \item AugUCB algorithm
    \item Theory
    \item Experiments
    \end{itemize}
\\\hline
    \end{tabular}
\end{frame}

%----------------------------------------------------------------------------------------
%	PRESENTATION SLIDES
%----------------------------------------------------------------------------------------


%\section{Introduction}
%\begin{frame}
\frametitle{Introduction}
\begin{itemize}
\item<1-> The bandit problem is a sequential decision making process where at each timestep we have to choose one action or arm from a set of arms.
\item<2-> There is a specific reward distribution attached to each arm.  After pulling an arm we receive a reward from the reward distribution specific to the arm. 
\item<3-> After say pulling each arm once, we are presented with an \emph{exploration-exploitation}  trade-off, that is whether to continue to pull the arm for which we have observed the highest estimated reward till now(exploitation) or to explore a new arm(exploration). 
\item<4-> If we become too greedy and always exploit, we may miss the chance of actually finding the optimal arm and get stuck with a sub-optimal arm.
\end{itemize}
\end{frame}

\begin{frame}
\frametitle{Some practical applications}
\begin{itemize}
\item<1-> Selecting the best channel (out of several existing channels) for mobile communications in a very short duration.
\item<2-> Selecting a small set of best workers (out of a very large pool of workers) whose productivity is above a threshold.
\item<3-> Selecting the best possible route for a message to pass through in a peer-to-peer network connection.
\end{itemize}
\end{frame}

\begin{frame}
\frametitle{Why study bandits at all?}
\begin{itemize}
\item<1-> We know of $\epsilon$-greedy \cite{sutton1998reinforcement} algorithm, we can simply stick to it.
\item<2-> But $\epsilon$-greedy only gives us an asymptotic guarantee. There is no guarantee that in a highly regressive environment how $\epsilon$-greedy will behave. Can we be better in our search?
\item<3-> Bandits allows us to study this behavior in a more formal way giving us strict guarantees regarding the performance of our algorithm.
%\item<4-> They form the linking pieces of a larger problem.
\item<4-> They are easy to implement.    
\end{itemize}
\end{frame}



\section{Stochastic Multi-Armed Bandit Problem}
\begin{frame}
\frametitle{Stochastic Multi-Armed Bandit Problem}
\begin{itemize}
\item<1-> In stochastic multi-armed bandit problem, we are presented with a finite set of actions or arms. 
\item<2-> The rewards for each of the arms drawn from distributions are identical and independent random variables. 
\item<3-> The learner does not know the mean of the distributions, denoted by $r_{i}$. 
\item<4-> The learner has to find the optimal arm the mean of whose distribution is denoted by $r^{*}$ such that $r^{*}> r_{i}, \forall i\in A$.
\item<5-> The distributions for each of the arms are fixed throughout the time horizon. 
\end{itemize}
\end{frame}

\begin{frame}
\frametitle{Basic Notations}
\begin{itemize}
\item<1-> Goal: To minimize Regret
\item<2-> Average reward of best action is $r^{*}$ and any other action $i$ as $r_{i}$. There are $K$ total actions. $T_{i}(n)$ is number of times tried action $i$ is executed till $n$-timesteps.
\item<3-> Cumulative Regret: The loss we suffer because of not pulling the optimal arm till the total number of timesteps  $T$. 
\begin{align*}
R_{T}=r^{*}T - \sum_{i\in A} r_{i}T_{i}(T),
\end{align*}
\item<4-> The expected regret of an algorithm after $T$ rounds can be written as
\begin{align*}
\E[R_{T}]= \sum_{i=1}^K \E[T_{i}(T)] \Delta_i,
\end{align*}
\item<4-> $\Delta_{i}=r^{*}-r_{i}$ denotes the gap between the means of the optimal arm and of the $i$-th arm. 
\end{itemize}
\end{frame}

\begin{frame}
\frametitle{Another Notion of Regret}
\begin{itemize}
\item<1-> Goal: To minimize Regret
\item<2-> Can we have a policy which achieves the minimum regret among all the possible environments available?
\item<3-> This is called the worst case gap-independent regret or sometimes called the minimax regret.
\item<4-> It is generally found by setting all the gaps to equal values of order $O\left( 1/\sqrt{T} \right)$.
\item<5-> Also we will define the hardness parameter $H$ as $H=\sum_{i=1}^{K}\dfrac{1}{\Delta_{i}^2}$
\end{itemize}
\end{frame}

\section{Problem Definition of SMAB}
\begin{frame}
\frametitle{Problem Definition of SMAB ({Chapter 2})}
\begin{itemize}
\item<1-> \textbf{Primary aim:} Minimize the cumulative regret by quickly identifying the arm whose expected mean is $r^*$ such that $r^* > r_i,\forall i\in\A$.
\item<2-> \textbf{Condition:} This has to be achieved within a finite $T$ timesteps.
\item<3-> The expected regret of an algorithm after $T$ timesteps is give by,
\begin{align*}
\E[R_{T}]= \sum_{i=1}^{K} \E[z_i (T)] \Delta_i,
\end{align*}
where $\Delta_{i}=r^{*}-r_{i}$ is the gap.
\end{itemize}
\end{frame}

%\begin{frame}
%\frametitle{Problem Definition of SMAB}
%\begin{itemize}
%\item<1-> We define the set $S_{\tau}=\lbrace i\in \mathcal{A}: r_{i}\geq \tau \rbrace$. 
%%Note that, $S_\tau$ is the set of all arms whose reward mean is greater than $\tau$. Let 
%\item<2-> $S_\tau^c$ denote the complement of $S_\tau$, i.e.,  $S_{\tau}^{c}=\lbrace i\in \mathcal{A}: r_{i} < \tau \rbrace$. 
%\item<3-> Let $\hat{S}_{\tau}$ denote the recommendation of a learning algorithm after $T$ time units of exploration, while $\hat{S}_{\tau}^c$ denotes its complement.
%
%%\item<4-> The performance of the learning agent is measured by the accuracy with which it can classify the arms into $S_{\tau}$ and $S_{\tau}^{c}$ after time horizon $T$. Equivalently, the \emph{loss} $\mathcal{L}(T)$ is defined as
%%\begin{align*}
%%\Ls (T) = \mathbb{I}\big(\lbrace S_{\tau}\cap \hat{S}_{\tau}^{c}\neq \emptyset\rbrace    \cup    \lbrace\hat{S}_{\tau}\cap S_{\tau}^{c}\neq \emptyset\rbrace\big).
%%\end{align*}			
%
%\item<4-> The goal of the learning agent is to minimize the expected loss:
%\begin{align*}
%\Ex[\Ls(T)] &= \Pb\big(\underbrace{\lbrace S_{\tau}\cap \hat{S}_{\tau}^{c} \neq \emptyset \rbrace}_{\textbf{Rejected good arms}}  \cup   \underbrace{\lbrace \hat{S}_{\tau}\cap S_{\tau}^{c} \neq \emptyset\rbrace}_{\textbf{Accepted bad arms}}\big) \\
%%& = 1 - \Pb\big(\lbrace \hat{S}_{\tau}\cap {S}_{\tau}^{c} = \emptyset \rbrace \cap \lbrace \hat{S}_{\tau}^c \cap {S}_{\tau} = \emptyset \rbrace \big )
%\end{align*}
%\end{itemize}
%\end{frame}



\section{Contributions in SMAB}
\begin{frame}
\frametitle{Contributions in SMAB ({Chapter 2})}
\begin{itemize}
\item<1-> We propose the Efficient-UCB-Variance (EUCBV) algorithm for the SMAB setting.
\item<2-> EUCBV takes into account the empirical variances of the arms along with mean estimates to quickly find the optimal arm.
\item<3-> It is the first variance-based arm elimination algorithm for the considered SMAB setting. 
\item<4-> It addresses an open problem discussed in {Auer and Ortner (2010)} of designing an algorithm that can eliminate arms based on variance estimates.
\item<5-> Theoretically it achieves an order-optimal regret bound, the first for an arm elimination algorithm in SMAB setting.
\item<6-> Empirically, it outperforms all the state-of-the-art algorithms for the considered environments.
%\item<5-> We also define a new problem complexity which uses empirical variance estimates along with arm's mean for giving the theoretical bound.
\end{itemize}
\end{frame}


\section{EUCBV Algorithm for SMAB}
\begin{frame}
\frametitle{EUCBV Algorithm for SMAB ({Chapter 3})}
\begin{figure}
%\caption{AugUCB Flowchart}
\includegraphics[scale=0.24]{img/EUCBV_flow.png}
\end{figure}
\end{frame}

\section{Theoretical Analysis of EUCBV}
\begin{frame}
\frametitle{Expected Regret of EUCBV ({Chapter 3})}
%
%\begin{theorem}
%For $T\geq K^{2.4}$, $\rho=\frac{1}{2}$ and $\psi=\frac{T}{K^2}$, the regret $R_T$ for EUCBV satisfies
%\begin{align*}
%\E [R_{T}] \leq &\sum\limits_{i\in \A :\Delta_{i} > b}\bigg\lbrace \dfrac{C_0 K^{4}}{T^{\frac{1}{4}}} + \bigg(\Delta_{i}+\dfrac{320\sigma_i^2\log{(\frac{T\Delta_{i}^{2}}{K})}}{\Delta_{i}}\bigg)\bigg \rbrace\\ 
%  & +\sum\limits_{i\in \A :0 < \Delta_{i}\leq b} \dfrac{C_2 K^{4}}{T^{\frac{1}{4}}} + \max_{i\in \A :0 < \Delta_{i}\leq b}\Delta_{i}T.
%\end{align*}
%
%for all $b\geq\sqrt{\frac{e}{T}}$ and $C_0, C_2$ are integer constants. 
%
%\end{theorem}

\begin{corollary}[\textbf{\textit{Gap-Independent Bound}}]
\label{Result:Corollary:1}
The regret of EUCBV is upper bounded by the following gap-independent expression:
\begin{align*}
	\E[R_{T}]\leq  \frac{C_3 K^5}{T^{\frac{1}{4}}} + 80\sqrt{KT}.
\end{align*}	
\end{corollary}


\begin{table}[b]
%\caption{AugUCB vs.\ State of the art}
\label{tab:comp-bds}
\begin{center}
\begin{tabular}{|p{1.5cm}|p{3.1cm}|p{3.1cm}|p{1.0cm}|}
% \toprule
\hline
Algorithm  & GD Bound & GI Bound & Var \\
% \midrule
\hline
%\hline
EUCBV      &$O\left( \frac{K\sigma_{\max}^{2}\log (\frac{T\Delta^2}{K})}{\Delta}\right)$ & $O\left(\sqrt{KT}\right)$ & Yes\\
%\hline
\hline
UCBV		&$O\left( \frac{K\sigma_{\max}^{2}\log T}{\Delta} \right)$ & $O\left(\sqrt{KT\log T}\right)$ & Yes\\
%\midrule
%\hline
\hline
MOSS         &$O\left( \frac{K^2\log (T\Delta^2 /K)}{\Delta}\right)$ & $O\left(\sqrt{KT}\right)$ & No\\
% \midrule
%\hline
\hline
OCUCB		&$O\left( \frac{K\log (T/ H_{i})}{\Delta}\right)$ &  $O\left(\sqrt{KT}\right)$ & No\\
%\midrule
\hline

%\bottomrule
\end{tabular}
\end{center}
\end{table}


%\begin{table}[tbp]
%%\caption{Regret upper bound of different algorithms}
%\label{tab:comp-bds}
%\begin{center}
%\begin{tabular}{p{3em}p{9em}p{7em}}
%\toprule
%Algorithm  &   \hspace*{1mm}Gap-Dependent & Gap-Independent \\
%\hline
%EUCBV		& $O\left( \dfrac{K\sigma_{\max}^{2}\log (\frac{T\Delta^2}{K})}{\Delta}\right)$ & $O\left(\sqrt{KT}\right)$\\
%%UCB1        & $O\left( \dfrac{K\log T}{\Delta} \right)$ & $O\left(\sqrt{KT\log T}\right)$ \\%\midrule
%UCBV        & $O\left( \dfrac{K\sigma_{\max}^{2}\log T}{\Delta} \right)$ & $O\left(\sqrt{KT\log T}\right)$ \\
%%UCB-Imp 		& $O\left( \dfrac{K\log (T\Delta^2)}{\Delta} \right)$ & $O\left(\sqrt{KT\log K}\right)$ \\%\midrule
%MOSS	     	& $O\left( \dfrac{K^2\log (T\Delta^2 /K)}{\Delta}\right)$ & $O\left(\sqrt{KT}\right)$\\%\midrule
%OCUCB     	& $O\left( \dfrac{K\log (T/ H_{i})}{\Delta}\right)$ & $O\left(\sqrt{KT}\right)$\\\midrule
%\end{tabular}
%\end{center}
%\vspace*{-2em}
%\end{table}



\end{frame}

\section{Experiments in SMAB}
\begin{frame}
\frametitle{Experiments in SMAB ({Chapter 3})}
\begin{figure}[tbp]
    \centering
    \begin{tabular}{cc}
    \subfigure[0.25\textwidth][Expt-$3$: Failure of TS]
    {
    		\pgfplotsset{
		tick label style={font=\Large},
		label style={font=\Large},
		legend style={font=\Large},
		ylabel style={yshift=14pt},
		}
        \begin{tikzpicture}[scale=0.45]
      	\begin{axis}[
		ylabel={Cumulative Regret},
		xlabel={timestep},
		grid=major,
        %clip mode=individual,grid,grid style={gray!30},
        clip=true,
        %clip mode=individual,grid,grid style={gray!30},
  		legend style={at={(0.5,1.2)},anchor=north, legend columns=3} ]
      	% UCB
		\addplot table{results1/NewExpt/Expt3/UCBV01_comp_subsampled.txt};
		\addplot table{results1/NewExpt/Expt3/EUCBV01_comp_subsampled.txt};
		\addplot table{results1/NewExpt/Expt3/MOSS01_comp_subsampled.txt};
		\addplot table{results1/NewExpt/Expt3/TS01_comp_subsampled.txt};
		\addplot table{results1/NewExpt/Expt3/OCUCB01_comp_subsampled.txt};
		\addplot table{results1/NewExpt/Expt3/BU01_comp_subsampled.txt};
      	\legend{UCBV,EUCBV,MOSS,TS-G,OCUCB,BU-G} 
      	\end{axis}
      	\end{tikzpicture}
  		\label{fig:3}
    }
    &
    \subfigure[0.25\textwidth][Expt-$4$: $3$ Group Variance]
    %with $r_{i_{{i}\neq {*}}}=0.05$ and $r^{*}=0.1$
    {
    	\pgfplotsset{
		tick label style={font=\Large},
		label style={font=\Large},
		legend style={font=\Large},
		ylabel style={yshift=14pt},
		}
        \begin{tikzpicture}[scale=0.45]
        \begin{axis}[
		xlabel={timestep},
		ylabel={Cumulative Regret},
        %clip mode=individual,grid,grid style={gray!30},
		grid=major,
		clip=true,
  		legend style={at={(0.5,1.2)},anchor=north, legend columns=3} ]
        % UCB
		\addplot table{results1/NewExpt/Expt41/UCBV01_comp_subsampled.txt};
		\addplot table{results1/NewExpt/Expt41/EUCBV01_comp_subsampled.txt};
		\addplot table{results1/NewExpt/Expt41/MOSS01_comp_subsampled.txt};
		\addplot table{results1/NewExpt/Expt41/TS01_comp_subsampled.txt};
		\addplot table{results1/NewExpt/Expt41/OCUCB01_comp_subsampled.txt};
		\addplot table{results1/NewExpt/Expt41/BU01_comp_subsampled.txt};
      	\legend{UCBV,EUCBV,MOSS,TS-G,OCUCB,BU-G} 
      	\end{axis}
        \end{tikzpicture}
        \label{fig:4}
    }
	\end{tabular}
	\label{fig:furtherExpt1}
    %\caption{Experiments with EUCBV}
\end{figure}
\end{frame}


%%%%%%%%%%%%%%%%%%%%%%%%%%%%%%%%%%%%%%%%%%%%%%%%%%%%%%%%%



\section{Problem Definition of TBP}
\input{ProbDefTBP}


\section{Contributions in TBP}
\input{contributionTBP}

%\section{Previous Works}
%

\begin{frame}
\frametitle{The Upper Confidence Bound (UCB) Approach}
\begin{itemize}
\item<1-> Since there is an initial uncertainty in the estimated mean ($\hat{r}_i$) introduce a confidence interval term $c_i$.
\item<2-> $c_i$ represents the uncertainty about $\hat{r}_i$. Higher the $c_i$, higher is the uncertainty.
\item<2-> $c_i$ ensures that the arm $i$ is properly explored and is gradually reduced with time as one pulls the arm $i$ more.
\item<3-> At every timestep pull arm $j\in \argmax_{i\in A} \lbrace \hat{r}_i + c_i\rbrace$ and this will ensure that proper exploration is done. 
\end{itemize}
\end{frame}

\begin{frame}
\frametitle{The UCB Approach}
\begin{figure}
%\caption{UCB Intuition}
\includegraphics[scale=0.3]{img/UCB_Drawing.png}
\end{figure}
\end{frame}


%\begin{frame}
%\frametitle{Previous Works (Pure Exploration)}
%\begin{itemize}
%\item<1-> The TBP problem also falls within the larger area called the Pure Exploration problem.
%\item<2-> In pure exploration problems the learner has to output a set of recommendations either with high confidence (fixed confidence) or after a specified number of rounds (fixed budget).
%\item<3-> Our considered TBP is a fixed budget pure exploration problem.
%\item<4-> Both APT and AugUCB reuses several ideas from Pure exploration problem. 
%\end{itemize}
%\end{frame}
%
%\begin{frame}
%\frametitle{Previous Works (Diagram)}
%\centering
%\begin{figure}
%\includegraphics[scale=0.3]{img/Settings2.png}
%\caption{TBP place within SMAB and Pure exploration}
%\end{figure}
%\end{frame}

%\begin{frame}
%\frametitle{Approach of UCB-Improved (UCB-Imp)}
%\begin{itemize}
%\item<1-> There is a strong relation between UCB-Imp \cite{auer2010ucb} and AugUCB where the former is used to find \emph{a single optimal arm as quickly as possible}.
%\item<2-> The basic idea of UCB-Improved is to divide the horizon into phases or rounds and initialize parameters.
%\item<3-> Pull all surviving arms equal number of times during a round.
%\item<4-> At the end of the round eliminate some sub-optimal arms (as judged by learner) based on elimination criteria.
%\item<5-> Reset parameters and proceed to next round.
%\end{itemize}
%\end{frame}
%
%\begin{frame}
%\frametitle{UCB-Improved (\cite{auer2010ucb})}
%\begin{algorithm}[H]
%\caption{UCB-Improved}
%\small
%\begin{algorithmic}[1]
%\State {\bf Input:} Time horizon $T$
%\State {\bf Initialization:} Set $B_{0}:=A$ and ${\epsilon}_{0}:=1$.
%\For{$m=0,1,..\big \lfloor \dfrac{1}{2}\log_{2} \dfrac{T}{e}\big\rfloor$}	
%\State Pull each arm in $B_m$, $n_{m}=\bigg\lceil\dfrac{2\log{( T{\epsilon}_{m}^{2})}}{{\epsilon}_{m}}\bigg\rceil$ number of times.
%%so that the total  it has been pulled is
%\ArmElim
%\State For each $i \in B_{m}$, delete arm ${i}$ from $B_{m}$ if,
%\begin{align*}
%\hat{r}_{i} + \sqrt{\dfrac{\log{(T{\epsilon}_{m}^{2})}}{2 n_{m}}}  < \max_{{j}\in B_{m}}\bigg\lbrace\hat{r}_{j} -\sqrt{\dfrac{\log{( T{\epsilon}_{m}^{2})}}{2 n_{m}}} \bigg\rbrace
%\end{align*}
%\EndArmElim
%%\ResParam
%\State Set ${\epsilon}_{m+1}:=\dfrac{{\epsilon}_{m}}{2}$, Set $B_{m+1}:=B_{m}$
%%\EndResParam
%\State Stop if $|B_{m}|=1$ and pull ${i}\in B_{m}$ till $T$ is reached.
%\EndFor
%\end{algorithmic}
%\end{algorithm}
%\end{frame}

\begin{frame}
\frametitle{Approach of UCB-Improved (UCB-Imp)}
\begin{figure}
%\caption{UCB Imp Approach}
\includegraphics[scale=0.25]{img/Ucb-Imp.png}
\end{figure}
\end{frame}



\begin{frame}
\frametitle{Intuition of UCB-Improved (UCB-Imp)}
\begin{figure}
%\caption{UCB Imp Intuition}
\includegraphics[scale=0.3]{img/Ucb_Imp_intuition.png}
\end{figure}
\end{frame}


\begin{frame}
\frametitle{APT Approach}
\begin{itemize}
\item<1-> The Anytime Parameter Free (APT) [{Locatelli et al. (2016)}] algorithm \textit{was proposed for TBP setting} in ICML 2016. 
\item<2-> This algorithm uses only mean estimation to find the $S_{\tau}$. 
\item<3-> Theoretically they proved this algorithm to be almost optimal when only mean estimation is used as a metric of comparison.
\item<4-> Empirically it outperformed other state-of-the-art algorithms which were modified to perform in the TBP setting.  
\end{itemize}
\end{frame}

\begin{frame}
\frametitle{APT Algorithm}
\begin{algorithm}[H]
\caption{APT}
\begin{algorithmic}
\State {\bf Input:} Time horizon $T$, threshold $\tau$, tolerance factor $\epsilon\geq 0$
\State Pull each arm once
\vspace{-3mm}
\State \For{$t=K+1,..,T$}
\State Pull arm $j\in\argmin_{i\in A}\Big\lbrace \left(|\hat{r}_{i} - \tau | + \epsilon\right)\sqrt{n_i}\Big\rbrace$ and observe the reward for arm $j$.
\EndFor
\State \textbf{Output:} $\hat{S}_{\tau}=\lbrace i: \hat{r}_{i}\geq \tau \rbrace$.
\end{algorithmic}
\end{algorithm}
\end{frame}


\begin{frame}
\frametitle{Intuition of APT}
\begin{figure}
%\caption{APT Intuition}
\includegraphics[scale=0.278]{img/APT_intuition.png}
\end{figure}
\end{frame}





%\begin{frame}
%\frametitle{Some technical details of UCB-Improved}
%\begin{itemize}
%\item<1-> We do not know the true means $r_i ,\forall i\in A$ of the distributions so we estimate it by the ${\epsilon}$ by initializing it from $1$.
%\item<2-> All rewards are assume to be bounded between $[0,1]$ and so $\Delta_{i} = (r^* - r_i)\in [0,1],\forall i\in A$ as well.
%\item<3-> UCB-Improved has fixed confidence interval  $c_{m}=\sqrt{\dfrac{\log{(T{\epsilon}_{m}^{2})}}{2 n_{m}}}$ for all arms in a particular round.
%\item<4-> $c_m$ ensures that whenever ${\epsilon}_{m}<\dfrac{\Delta_i}{2}$ in the $m$-th round, the arm $i$ gets eliminated.
%\end{itemize}
%\end{frame}

\section{AugUCB Algorithm for TBP}
\begin{frame}
\frametitle{AugUCB algorithm (Intuition, Arm pulling)}
\begin{itemize}
\item We define $\Delta_i = |r_i - \tau| $ . 
\item It is risky to eliminate arm $i$ while $\hat{r}_i$ is inside \emph{Margin}. 
\item Confidence interval $s_i$ will make sure arm $i$ is not eliminated while inside Margin with a high probability. 
\end{itemize}

\begin{figure}
\caption{AugUCB Intuition (Arm pulling)}
\includegraphics[scale=0.178]{img/SeminarThresholdBandit.png}
\end{figure}
\end{frame}

\begin{frame}
\frametitle{AugUCB algorithm (Intuition, Arm Elimination)}
\begin{itemize}
\item It is risky to eliminate arm $i$ while $\hat{r}_i$ is inside \emph{Margin}. 
\item Confidence interval $s_i$ will make sure arm $i$ is not eliminated while inside Margin with a high probability. 
\end{itemize}

\begin{figure}
\caption{AugUCB Intuition (Arm Elimination)}
\includegraphics[scale=0.178]{img/ArmElim1.png}
\end{figure}
\end{frame}

\begin{frame}
\frametitle{AugUCB algorithm (Intuition, Arm Elimination)}
\begin{figure}
\caption{AugUCB Intuition (Arm Elimination)}
\includegraphics[scale=0.178]{img/ArmElim2.png}
\end{figure}
\end{frame}


\begin{frame}
\frametitle{AugUCB algorithm}
\begin{itemize}
\item<1-> Like UCB-Imp, AugUCB also divides the time budget $T$ into rounds.
\item<2-> A crucial difference is that in every round instead of pulling all the arms equal number of times we pull the arm that minimizes $j\in\argmin_{i\in B_{m}}\Big\lbrace |\hat{r}_{i} - \tau | - 2s_{i}\Big\rbrace$ (like APT). 
\item<3-> At every timestep now we run the arm elimination check to eliminate sub-optimal arms.
\item<4-> At the end of the phase we reset the parameters. 
\item<5-> Note that the length of the phase, the exploration parameters and the confidence interval term $s_i  = \sqrt{\frac{\rho\psi_m (\hat{v}_{i}+1) \log ( T \epsilon_{m})}{4 n_{i}}}$ are set through detailed theoretical analysis. 
\end{itemize}
\end{frame}

\begin{frame}[allowframebreaks]
\frametitle{AugUCB algorithm}
%\begin{algorithm}[H]
%\caption{AugUCB}
%\label{alg:augucb}
\begin{algorithmic}
\State {\bf Input:} Time budget $T$; parameter $\rho$; threshold $\tau$
\State {\bf Initialization:} $B_{0}=\mathcal{A}$; $m=0$; $\epsilon_{0}=1$;
\begin{small}
\begin{align*}
M&=\left\lfloor \frac{1}{2}\log_{2} \frac{T}{e}\right\rfloor; 
\hspace{2mm}\psi_{0}=\frac{T\epsilon_{0}}{128\Big(\log(\frac{3}{16}K\log K)\Big)^2}; \\
\ell_{0}&=\left\lceil \frac{2\psi_0\log( T\epsilon_{0})}{\epsilon_{0}} \right\rceil ;
\hspace{2mm}N_{0}=K\ell_{0}
\end{align*}
\end{small}
\State Pull each arm once
\vspace{-3mm}
\State \For{$t=K+1,..,T$}
\State Pull arm $j\in\argmin_{i\in B_{m}}\Big\lbrace |\hat{r}_{i} - \tau | - 2s_{i}\Big\rbrace$
\vspace{-4.5mm}
\State \For{$i\in B_m$}
\vspace{-4.5mm}
\State \If{$(\hat{r}_{i} + s_i  < \tau - s_i)$ or $(\hat{r}_{i} - s_i > \tau + s_i)$}
\State $B_m\leftarrow B_m\backslash\{i\}$\hspace{4mm} (Arm deletion)
\EndIf
\EndFor
\vspace{-2mm}
\State \If{$t\geq N_{m}$ and $m \leq M$}
%\ResetParam
\State \textbf{Reset Parameters}
\State $\epsilon_{m+1}\leftarrow\frac{\epsilon_{m}}{2}$
\State $B_{m+1} \leftarrow B_{m}$
\State $\psi_{m+1}\leftarrow \frac{T\epsilon_{m+1}}{128(\log(\frac{3}{16}K\log K))^{2}}$
\State $\ell_{m+1}\leftarrow\left\lceil \frac{2\psi_{m+1}\log( T\epsilon_{m+1})}{\epsilon_{m+1}} \right\rceil$
\State $N_{m+1} \leftarrow t + |B_{m+1}|\ell_{m+1}$
\State $m \leftarrow m+1$
\EndIf
\EndFor
\State \textbf{Output:} $\hat{S}_{\tau}=\lbrace i: \hat{r}_{i}\geq \tau \rbrace$.
\end{algorithmic}
%\end{algorithm}
\end{frame}

\section{Theoretical Analysis of AugUCB}
\input{TheoryTBP}

\section{Experiments in TBP}
\input{exptTBP}

\section{Conclusion}
\begin{frame}
\frametitle{Conclusion ({Chapter 6})}
\begin{itemize}
\item<1-> We proposed the EUCBV algorithm for the SMAB setting which uses variance and mean estimation along with arm elimination to give an order-optimal theoretical guarantee.
\item<2-> We proposed the AugUCB algorithm for the fixed budget TBP  which uses variance estimation and arm elimination to give improved theoretical and experimental guarantees than mean estimation based algorithms.
\item<3-> Further studies are required to establish a lower bound on the expected loss of AugUCB.
\item<4-> A more detailed analysis of the non-uniform arm selection and parameter selection is also required for both AugUCB and EUCBV.
\end{itemize}
\end{frame}

\begin{frame}
\frametitle{Papers based on Thesis}
\begin{itemize}
\item Subhojyoti Mukherjee, K.P.~Naveen, Nandan Sudarsanam, and Balaraman Ravindran, "\textit{Thresholding Bandit with Augmented UCB}", {\em Proceedings of the Twenty-Sixth International Joint Conference on
               Artificial Intelligence, {IJCAI} 2017, Melbourne, Australia, August
               19-25, 2017,2515-2521}.
\item Subhojyoti Mukherjee, K.P.~Naveen, Nandan Sudarsanam, and Balaraman Ravindran, "\textit{Efficient UCBV: An Almost Optimal Algorithm using Variance Estimates}", {\em To appear in Proceedings of the Thirty-Second Association for the Advancement of Artificial Intelligence, {AAAI} 2018, New Orleans, Louisiana, USA, February 2-7}.
\end{itemize}
\end{frame}

%\section{References}
%\begin{frame}[allowframebreaks]
%\frametitle{References}
%%\bibliographystyle{named} 
%\bibliographystyle{plainnat} 
%%\bibliographystyle{dinat}
%\bibliography{ijcai17}
%\end{frame}


%------------------------------------------------

\begin{frame}
\Huge{\centerline{Thank You}}
\end{frame}

%----------------------------------------------------------------------------------------

\end{document} 