\begin{frame}
\frametitle{Problem Definition of TBP ({Chapter 4})}
\begin{itemize}
\item<1-> \textbf{Primary aim:} Identify all the arms whose mean of the reward distribution ($r_i$) is above a particular threshold $\tau$ given as input.
\item<2-> \textbf{Condition:} This has to be achieved within $T$ timesteps of exploration and this is termed as a fixed-budget problem.
%\item<3-> At the end of the given $T$ timesteps the learner must recommend a set of arms which (according to it) are the arms having reward mean above $\tau$.
\end{itemize}
\end{frame}

\begin{frame}
\frametitle{Problem Definition of TBP ({Chapter 4})}
\begin{itemize}
\item<1-> We define the set $S_{\tau}=\lbrace i\in \mathcal{A}: r_{i}\geq \tau \rbrace$. 
%Note that, $S_\tau$ is the set of all arms whose reward mean is greater than $\tau$. Let 
\item<2-> $S_\tau^c$ denote the complement of $S_\tau$, i.e.,  $S_{\tau}^{c}=\lbrace i\in \mathcal{A}: r_{i} < \tau \rbrace$. 
\item<3-> Let $\hat{S}_{\tau}$ denote the recommendation of a learning algorithm after $T$ time units of exploration, while $\hat{S}_{\tau}^c$ denotes its complement.

%\item<4-> The performance of the learning agent is measured by the accuracy with which it can classify the arms into $S_{\tau}$ and $S_{\tau}^{c}$ after time horizon $T$. Equivalently, the \emph{loss} $\mathcal{L}(T)$ is defined as
%\begin{align*}
%\Ls (T) = \mathbb{I}\big(\lbrace S_{\tau}\cap \hat{S}_{\tau}^{c}\neq \emptyset\rbrace    \cup    \lbrace\hat{S}_{\tau}\cap S_{\tau}^{c}\neq \emptyset\rbrace\big).
%\end{align*}			

\item<4-> The goal of the learning agent is to minimize the expected loss:
\begin{align*}
\Ex[\Ls(T)] &= \Pb\big(\underbrace{\lbrace S_{\tau}\cap \hat{S}_{\tau}^{c} \neq \emptyset \rbrace}_{\textbf{Rejected good arms}}  \cup   \underbrace{\lbrace \hat{S}_{\tau}\cap S_{\tau}^{c} \neq \emptyset\rbrace}_{\textbf{Accepted bad arms}}\big) \\
%& = 1 - \Pb\big(\lbrace \hat{S}_{\tau}\cap {S}_{\tau}^{c} = \emptyset \rbrace \cap \lbrace \hat{S}_{\tau}^c \cap {S}_{\tau} = \emptyset \rbrace \big )
\end{align*}
\end{itemize}
\end{frame}

%\begin{frame}
%\frametitle{Challenges in the TBP Settings}
%
%\begin{itemize}
%\item<1-> Closer the true mean of reward distribution of the arms' to the threshold $\Rightarrow$ harder the problem.
%\item<2-> Lesser the budget $\Rightarrow$ harder the problem.
%\item<3-> Higher the variance of reward distribution of the arms' $\Rightarrow$ harder the problem.
%
%\end{itemize}
%
%\end{frame}


%\begin{frame}
%\frametitle{Applications}
%\begin{itemize}
%\item<1-> Selecting the best channels (out of several existing channels) for mobile communications in a very short duration whose qualities are above an acceptable threshold (see Audibert and Bubeck (2010)).
%\item<2-> Selecting a small set of best workers (out of a very large pool of workers) whose productivity is above a threshold.
%\item<3-> In anomaly detection and classification (see {Locatelli et al. (2016)}).
%\end{itemize}
%\end{frame}

%The above TBP formulation has several applications, for instance, from areas ranging from anomaly detection and classification (see  \citet{locatelli2016optimal}) to industrial applications as well as in mobile communications (see \citet{audibert2010best})  where the users are to be allocated only those channels whose quality is above an acceptable threshold.