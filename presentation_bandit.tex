%%%%%%%%%%%%%%%%%%%%%%%%%%%%%%%%%%%%%%%%%
% Beamer Presentation
% LaTeX Template
% Version 1.0 (10/11/12)
%
% This template has been downloaded from:
% http://www.LaTeXTemplates.com
%
% License:
% CC BY-NC-SA 3.0 (http://creativecommons.org/licenses/by-nc-sa/3.0/)
%
%%%%%%%%%%%%%%%%%%%%%%%%%%%%%%%%%%%%%%%%%

%----------------------------------------------------------------------------------------
%	PACKAGES AND THEMES
%----------------------------------------------------------------------------------------

\documentclass{beamer}


\mode<presentation> {

% The Beamer class comes with a number of default slide themes
% which change the colors and layouts of slides. Below this is a list
% of all the themes, uncomment each in turn to see what they look like.

%\usetheme{default}
%\usetheme{AnnArbor}
%\usetheme{Antibes}
%\usetheme{Bergen}
%\usetheme{Berkeley}
%\usetheme{Berlin}
%\usetheme{Boadilla}
%\usetheme{CambridgeUS}
%\usetheme{Copenhagen}
%\usetheme{Darmstadt}
%\usetheme{Dresden}
%\usetheme{Frankfurt}
%\usetheme{Goettingen}
%\usetheme{Hannover}
%\usetheme{Ilmenau}
%\usetheme{JuanLesPins}
%\usetheme{Luebeck}
\usetheme{Madrid}
%\usetheme{Malmoe}
%\usetheme{Marburg}
%\usetheme{Montpellier}
%\usetheme{PaloAlto}
%\usetheme{Pittsburgh}
%\usetheme{Rochester}
%\usetheme{Singapore}
%\usetheme{Szeged}
%\usetheme{Warsaw}

% As well as themes, the Beamer class has a number of color themes
% for any slide theme. Uncomment each of these in turn to see how it
% changes the colors of your current slide theme.

%\usecolortheme{albatross}
%\usecolortheme{beaver}
%\usecolortheme{beetle}
%\usecolortheme{crane}
%\usecolortheme{dolphin}
%\usecolortheme{dove}
%\usecolortheme{fly}
%\usecolortheme{lily}
%\usecolortheme{orchid}
%\usecolortheme{rose}
%\usecolortheme{seagull}
%\usecolortheme{seahorse}
%\usecolortheme{whale}
%\usecolortheme{wolverine}

%\setbeamertemplate{footline} % To remove the footer line in all slides uncomment this line
%\setbeamertemplate{footline}[page number] % To replace the footer line in all slides with a simple slide count uncomment this line

%\setbeamertemplate{navigation symbols}{} % To remove the navigation symbols from the bottom of all slides uncomment this line
}

\usepackage{macros}

%----------------------------------------------------------------------------------------
%	TITLE PAGE
%----------------------------------------------------------------------------------------

\title[Tutorial on Bandit]{Tutorial on Bandit} % The short title appears at the bottom of every slide, the full title is only on the title page

\author{Subhojyoti Mukherjee} % Your name
\institute[IIT Madras] % Your institution as it will appear on the bottom of every slide, may be shorthand to save space
{
IIT Madras \\ % Your institution for the title page
\medskip
%\textit{john@smith.com} % Your email address
}
\date{\today} % Date, can be changed to a custom date

\begin{document}
\nocite{*}
\begin{frame}
\titlepage % Print the title page as the first slide
\end{frame}

\begin{frame}
\frametitle{Overview} % Table of contents slide, comment this block out to remove it
\tableofcontents % Throughout your presentation, if you choose to use \section{} and \subsection{} commands, these will automatically be printed on this slide as an overview of your presentation
\end{frame}

%----------------------------------------------------------------------------------------
%	PRESENTATION SLIDES
%----------------------------------------------------------------------------------------


\section{Introduction}
\begin{frame}
\frametitle{Introduction to bandits}
\begin{itemize}
\item<1-> The bandit problem is a sequential decision making process where at each timestep we have to choose one action or arm from a set of arms.
\item<2-> There is a specific reward distribution attached to each arm.  After pulling an arm we receive a reward (without delay) from the reward distribution specific to the arm. 
\item<3-> After (say) pulling each arm once, we are presented with an \emph{exploration-exploitation}  trade-off, that is whether to continue to pull the arm for which we have observed the highest estimated reward till now(exploitation) or to explore a new arm(exploration). 
\item<4-> If we become too greedy and always exploit, we may miss the chance of actually finding the optimal arm and get stuck with a sub-optimal arm.
\end{itemize}
\end{frame}




\section{Stochastic Multi-Armed Bandit Problem}
\begin{frame}
\frametitle{Stochastic Multi-Armed Bandit Problem (SMAB)}
\begin{itemize}
\item<1-> The thresholding bandit problem falls under the broad area of stochastic multi-armed bandit problem.
\item<2-> In SMAB problem, we are presented with a finite set of actions or arms belonging to set $A$ such that $|A|=K$. 
\item<3-> The rewards for each of the arms are identical and independent random variables drawn from distribution specific to the arm.
\item<4-> The learner does not know the mean of the distributions, denoted by $r_{i},\forall i\in A$ or  the variance denoted by $\sigma_i^2$. 
\item<5-> The distributions for each of the arms are fixed throughout the time horizon denoted by $T$. 
\end{itemize}
\end{frame}




\section{Literature Survey}
\begin{frame}
\frametitle{Literature Survey}
\begin{itemize}
\item<1-> A considerable amount of research has been conducted on SMAB. We can divide the gamut of literature on SMAB into broadly two categories:
\begin{itemize}
\item<2-> \textbf{Frequentist approach:} In this approach for each of the arms compute a dynamic allocation index that depends only on the number of draws on the arm and exploration parameters and choose the arm with the maximal index. Eg:  \emph{UCB} variants
\item<3-> \textbf{Bayesian Approach:} In this approach we start with a prior guess over the performance of each of the arms. Then we pull an arm by behaving greedily based on our guess, receive the reward and then we update our prior for the arm. Eg: TS 
\end{itemize} 
\item<4-> We will be focusing on UCB based approaches in our work.
\end{itemize}
\end{frame}

\begin{frame}
\frametitle{Literature Survey}
\begin{itemize}
\item<1-> The literature under UCB can be broadly classified into three categories:
\begin{itemize}
\item<2-> \textbf{Mean-based estimation: } In this approach at every timestep we choose an arm based on $\hat{r}_i$ and its confidence interval $c_i$. Eg: UCB1 \cite{auer2002finite}, MOSS \cite{audibert2009minimax}, UCB-Improved \cite{auer2010ucb}
\item<3-> \textbf{Mean and Variance based Estimation: } Here, at every timestep we choose an arm based on $\hat{r}_i$, $\hat{V}_i$ and its confidence interval $c_i$. Eg: UCB-Normal \cite{auer2002finite}l, UCB-V \cite{audibert2009exploration}.
%\item<4-> \textbf{Divergence based methods: } In this approach at every timestep we choose an arm that Eg: KL-UCB \cite{garivier2011kl}, DMED \cite{honda2010asymptotically}.
\end{itemize}
\end{itemize} 
\end{frame}

\begin{frame}
\frametitle{UCB1 Algorithm (\cite{auer2002finite})}
\begin{algorithm}[H]
\caption{UCB1}
\begin{algorithmic}[1]
\State Pull each arm once
 \For{$t=K+1,..., T$}
\State Pull the arm such that $\max_{i\in A}\bigg\lbrace\hat{r}_{i} + \sqrt{\dfrac{2\log t}{s_i}}\bigg\rbrace$
\State $t:=t+1 $
 \EndFor
\end{algorithmic}
\end{algorithm}

\begin{itemize}
\item<1-> Maintain an upper confidence bound ($c_i$) for each of the arms
\item<1-> This $c_i$ will help in sufficiently exploring sub-optimal arms and then exploiting the optimal arm.
\item<1-> The gap-independent regret bound of $O\left( \sqrt{KT\log T}\right) $ and gap-dependent bound of $O\left( \dfrac{K \log (T)}{\Delta} \right)$.
\end{itemize}
\end{frame}

\begin{frame}
\frametitle{Minimax Optimal Strategy in the Stochastic Case (\cite{audibert2009minimax})}
\begin{algorithm}[H]
\caption{MOSS}
\begin{algorithmic}[1]
\State Pull each arm once
 \For{$t=K+1,..., T$}
\State Pull the arm such that $\max_{i\in A}\bigg\lbrace\hat{r}_{i} + \sqrt{\dfrac{\max\lbrace 0,\log(\frac{T}{K s_i})\rbrace}{s_i}}\bigg\rbrace$
\State $t:=t+1 $
 \EndFor
\end{algorithmic}
\end{algorithm}
\begin{itemize}
\item<1-> UCB1 suffers from a worst case regret of $O\left( \sqrt{KT\log T }\right) $.
\item<1-> MOSS corrects this and gives us a gap-independent regret bound of $O\left( \sqrt{KT}\right)$ and gap-dependent bound of $O\left( \dfrac{K^2 \log (\frac{T\Delta^2}{K})}{\Delta}\right)$.
%UCB1 performs badly when the gaps are closer to $\sqrt{\dfrac{K}{T}}$.
\end{itemize}
\end{frame}

\begin{frame}
\frametitle{Approach of UCB-Improved}
\begin{itemize}
\item<1-> The basic idea of UCB-Improved is to divide the horizon into phases or rounds and initialize parameters.
\item<2-> Pull all surviving arms equal number of times during a round.
\item<3-> At the end of the round eliminate some arms based on some criteria.
\item<4-> Reset parameters and proceed to next round.
\item<5-> UCB-Imp achieves a gap-independent regret bound of $O\left( \sqrt{KT\log K}\right)$ and gap-dependent bound of $O\left( \dfrac{K \log (T\Delta^2)}{\Delta}\right)$.
\end{itemize}
\end{frame}

\begin{frame}
\frametitle{UCB-Improved (\cite{auer2010ucb})}
\begin{algorithm}[H]
\caption{UCB-Improved}
\begin{algorithmic}[1]
\State {\bf Input:} Time horizon $T$
\State {\bf Initialization:} Set $B_{0}:=A$ and $\tilde{\Delta}_{0}:=1$.
\For{$m=0,1,..\big \lfloor \dfrac{1}{2}\log_{2} \dfrac{T}{e}\big\rfloor$}	
\State Pull each arm in $B_m$, $n_{m}=\bigg\lceil\dfrac{2\log{( T\tilde{\Delta}_{m}^{2})}}{\tilde{\Delta}_{m}}\bigg\rceil$ number of times.
%so that the total  it has been pulled is
\ArmElim
\State For each $i \in B_{m}$, delete arm ${i}$ from $B_{m}$ if,
\begin{align*}
\bar{X}_{i} + \sqrt{\dfrac{\log{(T\tilde{\Delta}_{m}^{2})}}{2 n_{m}}}  < \max_{{j}\in B_{m}}\bigg\lbrace\bar{X}_{j} -\sqrt{\dfrac{\log{( T\tilde{\Delta}_{m}^{2})}}{2 n_{m}}} \bigg\rbrace
\end{align*}
\EndArmElim
%\ResParam
\State Set $\tilde{\Delta}_{m+1}:=\dfrac{\tilde{\Delta}_{m}}{2}$, Set $B_{m+1}:=B_{m}$
%\EndResParam
\State Stop if $|B_{m}|=1$ and pull ${i}\in B_{m}$ till $n$ is reached.
\EndFor
\end{algorithmic}
\end{algorithm}
\end{frame}

\begin{frame}
\frametitle{Some technical details of UCB-Improved}
\begin{itemize}
\item<1-> We do not know the true means $\mu_i ,\forall i\in A$ of the distributions so we estimate it by the $\tilde{\Delta}$ by initializing it from $1$.
\item<2-> All rewards are assume to be bounded between $[0,1]$ and so $\Delta_{i}\in [0,1],\forall i\in A$ as well.
\item<3-> As opposed to UCB1, MOSS and OCUCB, UCB-Improved has fixed confidence interval  $c_{m}=\sqrt{\dfrac{\log{(T\tilde{\Delta}_{m}^{2})}}{2 n_{m}}}$ for all arms in a particular phase.
\item<4-> $c_m$ ensures that whenever $\tilde{\Delta}_{m}<\dfrac{\Delta_i}{2}$ in the $m$-th round, the arm $i$ gets eliminated.
\end{itemize}
\end{frame}

\begin{frame}
\frametitle{Optimally confident UCB (\cite{lattimore2015optimally})}
\begin{algorithm}[H]
\caption{MOSS}
\begin{algorithmic}[1]
\State \textbf{Input: } K,T, $\alpha$, $\psi$
\State Pull each arm once
 \For{$t=K+1,..., T$}
\State Pull the arm such that $\max_{i\in A}\bigg\lbrace\hat{r}_{i} + \sqrt{\alpha\dfrac{\max\lbrace 0,\log(\frac{\psi T}{ s_i})\rbrace}{s_i}}\bigg\rbrace$
\State $t:=t+1 $
 \EndFor
\end{algorithmic}
\end{algorithm}
\begin{itemize}
\item<1-> UCB1 is too conservative in exploiting, MOSS is not conservative enough and tends to explore more often than required. 
\item<1-> OCUCB correctly balances this and achieves a gap-independent regret bound  $O\left(\sqrt{KT}\right)$ and gap-dependent bound $O \left(\dfrac{K\log (T/H)}{\Delta}\right)$.
\end{itemize}
\end{frame}

\begin{frame}
\frametitle{Comparison of UCB1, MOSS, OCUCB, UCB-Improved}
\begin{table}
\caption{Cumulative Regret of Algorithms}
\begin{center}
\begin{tabular}{|c|c|}
\toprule
Algorithm  & Upper bound on Cumulative Regret\\
\midrule
UCB1        &$\min\left \lbrace O\left(\dfrac{K\log T}{\Delta} \right), O\left( \sqrt{KT\log T}\right) \right\rbrace$ \\\midrule
MOSS        &$\min\left \lbrace O\left(\dfrac{K^2\log (T\Delta^2/K)}{\Delta} \right), O\left( \sqrt{KT}\right) \right\rbrace$ \\\midrule
OCUCB        &$\min\left \lbrace O\left(\dfrac{K\log (T/H)}{\Delta} \right), O\left( \sqrt{KT}\right) \right\rbrace$ \\\midrule
UCB-Improved      &$\min\left \lbrace O\left(\dfrac{K\log (T\Delta^2)}{\Delta} \right), O\left( \sqrt{KT\log K}\right) \right\rbrace$\\\bottomrule
\end{tabular}
\end{center}
\end{table}
\end{frame}



\section{Algorithms}
\begin{frame}
\frametitle{UCB1 Algorithm (\cite{auer2002finite})}
\begin{algorithm}[H]
\caption{UCB1}
\begin{algorithmic}[1]
\State Pull each arm once
 \For{$t=K+1,..., T$}
\State Pull the arm such that $\max_{i\in A}\bigg\lbrace\hat{r}_{i} + \sqrt{\dfrac{2\log t}{s_i}}\bigg\rbrace$
\State $t:=t+1 $
 \EndFor
\end{algorithmic}
\end{algorithm}

\begin{itemize}
\item<1-> Maintain an upper confidence bound ($c_i$) for each of the arms
\item<1-> This $c_i$ will help in sufficiently exploring sub-optimal arms and then exploiting the optimal arm.
\item<1-> The gap-independent regret bound of $O\left( \sqrt{KT\log T}\right) $ and gap-dependent bound of $O\left( \dfrac{K \log (T)}{\Delta} \right)$.
\end{itemize}
\end{frame}

\begin{frame}
\frametitle{Minimax Optimal Strategy in the Stochastic Case (\cite{audibert2009minimax})}
\begin{algorithm}[H]
\caption{MOSS}
\begin{algorithmic}[1]
\State Pull each arm once
 \For{$t=K+1,..., T$}
\State Pull the arm such that $\max_{i\in A}\bigg\lbrace\hat{r}_{i} + \sqrt{\dfrac{\max\lbrace 0,\log(\frac{T}{K s_i})\rbrace}{s_i}}\bigg\rbrace$
\State $t:=t+1 $
 \EndFor
\end{algorithmic}
\end{algorithm}
\begin{itemize}
\item<1-> UCB1 suffers from a worst case regret of $O\left( \sqrt{KT\log T }\right) $.
\item<1-> MOSS corrects this and gives us a gap-independent regret bound of $O\left( \sqrt{KT}\right)$ and gap-dependent bound of $O\left( \dfrac{K^2 \log (\frac{T\Delta^2}{K})}{\Delta}\right)$.
%UCB1 performs badly when the gaps are closer to $\sqrt{\dfrac{K}{T}}$.
\end{itemize}
\end{frame}

\begin{frame}
\frametitle{Approach of UCB-Improved}
\begin{itemize}
\item<1-> The basic idea of UCB-Improved is to divide the horizon into phases or rounds and initialize parameters.
\item<2-> Pull all surviving arms equal number of times during a round.
\item<3-> At the end of the round eliminate some arms based on some criteria.
\item<4-> Reset parameters and proceed to next round.
\item<5-> UCB-Imp achieves a gap-independent regret bound of $O\left( \sqrt{KT\log K}\right)$ and gap-dependent bound of $O\left( \dfrac{K \log (T\Delta^2)}{\Delta}\right)$.
\end{itemize}
\end{frame}

\begin{frame}
\frametitle{UCB-Improved (\cite{auer2010ucb})}
\begin{algorithm}[H]
\caption{UCB-Improved}
\begin{algorithmic}[1]
\State {\bf Input:} Time horizon $T$
\State {\bf Initialization:} Set $B_{0}:=A$ and $\tilde{\Delta}_{0}:=1$.
\For{$m=0,1,..\big \lfloor \dfrac{1}{2}\log_{2} \dfrac{T}{e}\big\rfloor$}	
\State Pull each arm in $B_m$, $n_{m}=\bigg\lceil\dfrac{2\log{( T\tilde{\Delta}_{m}^{2})}}{\tilde{\Delta}_{m}}\bigg\rceil$ number of times.
%so that the total  it has been pulled is
\ArmElim
\State For each $i \in B_{m}$, delete arm ${i}$ from $B_{m}$ if,
\begin{align*}
\bar{X}_{i} + \sqrt{\dfrac{\log{(T\tilde{\Delta}_{m}^{2})}}{2 n_{m}}}  < \max_{{j}\in B_{m}}\bigg\lbrace\bar{X}_{j} -\sqrt{\dfrac{\log{( T\tilde{\Delta}_{m}^{2})}}{2 n_{m}}} \bigg\rbrace
\end{align*}
\EndArmElim
%\ResParam
\State Set $\tilde{\Delta}_{m+1}:=\dfrac{\tilde{\Delta}_{m}}{2}$, Set $B_{m+1}:=B_{m}$
%\EndResParam
\State Stop if $|B_{m}|=1$ and pull ${i}\in B_{m}$ till $n$ is reached.
\EndFor
\end{algorithmic}
\end{algorithm}
\end{frame}

\begin{frame}
\frametitle{Some technical details of UCB-Improved}
\begin{itemize}
\item<1-> We do not know the true means $\mu_i ,\forall i\in A$ of the distributions so we estimate it by the $\tilde{\Delta}$ by initializing it from $1$.
\item<2-> All rewards are assume to be bounded between $[0,1]$ and so $\Delta_{i}\in [0,1],\forall i\in A$ as well.
\item<3-> As opposed to UCB1, MOSS and OCUCB, UCB-Improved has fixed confidence interval  $c_{m}=\sqrt{\dfrac{\log{(T\tilde{\Delta}_{m}^{2})}}{2 n_{m}}}$ for all arms in a particular phase.
\item<4-> $c_m$ ensures that whenever $\tilde{\Delta}_{m}<\dfrac{\Delta_i}{2}$ in the $m$-th round, the arm $i$ gets eliminated.
\end{itemize}
\end{frame}

\begin{frame}
\frametitle{Optimally confident UCB (\cite{lattimore2015optimally})}
\begin{algorithm}[H]
\caption{MOSS}
\begin{algorithmic}[1]
\State \textbf{Input: } K,T, $\alpha$, $\psi$
\State Pull each arm once
 \For{$t=K+1,..., T$}
\State Pull the arm such that $\max_{i\in A}\bigg\lbrace\hat{r}_{i} + \sqrt{\alpha\dfrac{\max\lbrace 0,\log(\frac{\psi T}{ s_i})\rbrace}{s_i}}\bigg\rbrace$
\State $t:=t+1 $
 \EndFor
\end{algorithmic}
\end{algorithm}
\begin{itemize}
\item<1-> UCB1 is too conservative in exploiting, MOSS is not conservative enough and tends to explore more often than required. 
\item<1-> OCUCB correctly balances this and achieves a gap-independent regret bound  $O\left(\sqrt{KT}\right)$ and gap-dependent bound $O \left(\dfrac{K\log (T/H)}{\Delta}\right)$.
\end{itemize}
\end{frame}

\begin{frame}
\frametitle{Comparison of UCB1, MOSS, OCUCB, UCB-Improved}
\begin{table}
\caption{Cumulative Regret of Algorithms}
\begin{center}
\begin{tabular}{|c|c|}
\toprule
Algorithm  & Upper bound on Cumulative Regret\\
\midrule
UCB1        &$\min\left \lbrace O\left(\dfrac{K\log T}{\Delta} \right), O\left( \sqrt{KT\log T}\right) \right\rbrace$ \\\midrule
MOSS        &$\min\left \lbrace O\left(\dfrac{K^2\log (T\Delta^2/K)}{\Delta} \right), O\left( \sqrt{KT}\right) \right\rbrace$ \\\midrule
OCUCB        &$\min\left \lbrace O\left(\dfrac{K\log (T/H)}{\Delta} \right), O\left( \sqrt{KT}\right) \right\rbrace$ \\\midrule
UCB-Improved      &$\min\left \lbrace O\left(\dfrac{K\log (T\Delta^2)}{\Delta} \right), O\left( \sqrt{KT\log K}\right) \right\rbrace$\\\bottomrule
\end{tabular}
\end{center}
\end{table}
\end{frame}



\section{Problem Definition}
\begin{frame}
\frametitle{Problem Definition}
\begin{itemize}
\item<1-> The primary aim in the thresholding bandit problem (TBP) is to identify the arms whose mean of the reward distribution is above a particular threshold $\tau$ given as input.
\item<2-> The above goal has to be achieved within $T$ timesteps of exploration and this is termed as a fixed-budget problem.
\item<3-> At the end of the given $T$ timesteps the learner must recommend a set of arms which (according to it) are the arms having reward mean above $\tau$.
\end{itemize}
\end{frame}

\begin{frame}
\frametitle{Problem Definition}
\begin{itemize}
\item<1-> We define the set $S_{\tau}=\lbrace i\in \mathcal{A}: r_{i}\geq \tau \rbrace$. 
%Note that, $S_\tau$ is the set of all arms whose reward mean is greater than $\tau$. Let 
\item<2-> $S_\tau^c$ denote the complement of $S_\tau$, i.e.,  $S_{\tau}^{c}=\lbrace i\in \mathcal{A}: r_{i} < \tau \rbrace$. 
\item<3-> Let $\hat{S}_{\tau}$ denote the recommendation of a learning algorithm after $T$ time units of exploration, while $\hat{S}_{\tau}^c$ denotes its complement.

%\item<4-> The performance of the learning agent is measured by the accuracy with which it can classify the arms into $S_{\tau}$ and $S_{\tau}^{c}$ after time horizon $T$. Equivalently, the \emph{loss} $\mathcal{L}(T)$ is defined as
%\begin{align*}
%\Ls (T) = \mathbb{I}\big(\lbrace S_{\tau}\cap \hat{S}_{\tau}^{c}\neq \emptyset\rbrace    \cup    \lbrace\hat{S}_{\tau}\cap S_{\tau}^{c}\neq \emptyset\rbrace\big).
%\end{align*}			

\item<4-> The goal of the learning agent is to minimize the expected loss:
\begin{align*}
\Ex[\Ls(T)] &= \Pb\big(\lbrace S_{\tau}\cap \hat{S}_{\tau}^{c} \neq \emptyset \rbrace  \cup   \lbrace \hat{S}_{\tau}\cap S_{\tau}^{c} \neq \emptyset\rbrace\big) \\
& = 1 - \Pb\big(\lbrace \hat{S}_{\tau}\cap {S}_{\tau}^{c} = \emptyset \rbrace \cap \lbrace \hat{S}_{\tau}^c \cap {S}_{\tau} = \emptyset \rbrace \big )
\end{align*}
\end{itemize}
\end{frame}

\begin{frame}
\frametitle{Challenges in the TBP Settings}
\begin{itemize}
\item<1-> The more number of arms' means are closer to the threshold the harder is to discriminate between them.
\item<2-> The lesser the budget is, the harder the problem becomes.
\item<3-> The higher the variance of the arms' the more difficult is to discriminate.
\end{itemize}
\end{frame}


\begin{frame}
\frametitle{Some practical applications}
\begin{itemize}
\item<1-> Selecting the best channels (out of several existing channels) for mobile communications in a very short duration whose qualities are above an acceptable threshold (see \cite{audibert2010best}).
\item<2-> Selecting a small set of best workers (out of a very large pool of workers) whose productivity is above a threshold.
\item<3-> In anomaly detection and classification (see {Locatelli et al. (2016)}).
\end{itemize}
\end{frame}

%The above TBP formulation has several applications, for instance, from areas ranging from anomaly detection and classification (see  \citet{locatelli2016optimal}) to industrial applications as well as in mobile communications (see \citet{audibert2010best})  where the users are to be allocated only those channels whose quality is above an acceptable threshold.

\section{Clustered UCB}
\begin{frame}
\frametitle{Approach of ClusUCB}
\begin{itemize}
\item<1-> The basic idea of ClusUCB directly follows from UCB-Imp. It starts by dividing the horizon into rounds and initializing parameters.
\item<2-> It then creates $p$ fixed clusters (given as an input) and randomly assign arms into each of them such that each cluster contains equal number of arms.
\item<3-> Pull all surviving arms equal number of times during a round.
\item<4-> At the end of the round eliminate arms inside each cluster by comparing its performance against the best arm in the cluster. 
\item<5-> Also eliminate clusters with all of its arms by comparing its performance against the globally  best arm. 
\item<5-> Reset parameters and move to the next round.
\item<6-> At a higher level ClusUCB behaves like $p$ independently running UCB-Imp with the exploration parameters $\rho_a,\rho_s$ and $\psi$ helping in overcoming early exploration.
\end{itemize}
\end{frame}

\begin{frame}[allowframebreaks]
\frametitle{Approach of ClusUCB}
%\begin{algorithm}[H]
%\caption{ClusUCB}
%\label{alg:clusucb}
\begin{algorithmic}[1]
\State {\bf Input:} Number of clusters $p$, time horizon $T$, exploration parameters $\rho_a$, $\rho_s$ and $\psi$.
\State {\bf Initialization:} Set $B_{0}:=A$, $S_0 = S$ and $\epsilon_{0}:=1$.
\State Create a partition $S_0$ of the arms at random into $p$ clusters of size up to $\ell=\bigg\lceil \frac{K}{p} \bigg\rceil$ each.
\For{$m=0,1,..\big \lfloor \frac{1}{2}\log_{2} \frac{7T}{K}\big\rfloor$}	
\State Pull each arm in $B_m$ so that the total number of times it has been pulled is $n_{m}=\bigg\lceil\frac{2\log{(\psi T\epsilon_{m}^{2})}}{\epsilon_{m}}\bigg\rceil$. 
\ArmElim
\State For each cluster $s_k \in S_{m}$, delete arm ${i}\in s_{k}$ from $B_{m}$ if
\begin{align*}
\hat{r}_{i} + \sqrt{\frac{\rho_{a}\log{(\psi T\epsilon_{m}^{2})}}{2 n_{m}}}  < \max_{{j}\in s_{k}}\bigg\lbrace\hat{r}_{j} -\sqrt{\frac{\rho_{a}\log{(\psi T\epsilon_{m}^{2})}}{2 n_{m}}} \bigg\rbrace
\end{align*}
% where $\rho_{a}=\frac{1}{w_{m}}$ and remove all such arms from $B_{m}$.
\EndArmElim
\ClusElim
\State Delete cluster $s_{k}\in S_{m}$ and remove all arms $i\in s_{k}$ from $B_{m}$ if 
\begin{align*}
 \max_{{i}\in s_{k}}\bigg\lbrace\hat{r}_{i} + \sqrt{\frac{\rho_{s}\log{(\psi T\epsilon_{m}^{2})}}{2 n_{m}}}\bigg\rbrace 
 < \max_{{j}\in B_{m}} \bigg\lbrace\hat{r}_{j} - \sqrt{\frac{\rho_{s} \log{(\psi T\epsilon_{m}^{2})}}{2 n_{m}}}\bigg\rbrace.
\end{align*}
%  and remove all such arms in the cluster $s_{k}$ from $B_{m}$ to obtain $B_{m+1}$.
\EndClusElim
\State Set $\epsilon_{m+1}:=\frac{\epsilon_{m}}{2}$\vspace{0.5ex}
\State Set $B_{m+1}:=B_{m}$
%\State \hspace*{2em} $\ell_{m+1}:=\min\lbrace 2\ell_{m}, K\rbrace$
%\State \hspace*{2em} $w_{m+1}:=2w_{m}$
\State Stop if $|B_{m}|=1$ and pull ${i}\in B_{m}$ till $T$ is reached.
\EndFor
\end{algorithmic}
%\end{algorithm}
\end{frame}

\begin{frame}
\frametitle{UCB1, MOSS, OCUCB, UCB-Imp, ClusUCB}
\begin{table}
%\caption{Cumulative Regret of Algorithms}
\begin{center}
\begin{tabular}{|c|c|}
\toprule
Algorithm  & Upper bound on Cumulative Regret\\
\midrule
UCB1        &$\min\left \lbrace O\left(\dfrac{K\log T}{\Delta} \right), O\left( \sqrt{KT\log T}\right) \right\rbrace$ \\\midrule
MOSS        &$\min\left \lbrace O\left(\dfrac{K^2\log (T\Delta^2/K)}{\Delta} \right), O\left( \sqrt{KT}\right) \right\rbrace$ \\\midrule
OCUCB        &$\min\left \lbrace O\left(\dfrac{K\log (T/H)}{\Delta} \right), O\left( \sqrt{KT}\right) \right\rbrace$ \\\midrule
UCB-Improved      &$\min\left \lbrace O\left(\dfrac{K\log (T\Delta^2)}{\Delta} \right), O\left( \sqrt{KT\log K}\right) \right\rbrace$\\\midrule
ClusUCB      &$\min\left \lbrace O\left(\dfrac{K\log (T\Delta^2 /\sqrt{\log K})}{\Delta} \right), O\left( \sqrt{KT\log K}\right) \right\rbrace$\\\bottomrule
\end{tabular}
\end{center}
\end{table}
\end{frame}

\section{Efficient Clustered UCB}
\begin{frame}
\frametitle{Approach of EClusUCB}
\begin{itemize}
\item<1-> One of the main problems of ClusUCB is that it's still a round based algorithm.
\item<2-> In every round it pulls all the arms equal number of times, which although is less compared to UCB-Improved but still we can be better.
\item<3-> One simple solution is to pull the arm with the highest UCB at every timestep. This is called optimistic greedy sampling for UCB-Imp (see \cite{liu2016modification}).
\item<4-> We introduce this in Efficient ClusUCB or EClusUCB.
\end{itemize}
\end{frame}

\begin{frame}
\frametitle{Technical Details of EClusUCB}
\begin{itemize}
\item<1-> EClusUCB has three exploration parameters $\rho_a$, $\rho_s$ and $\psi$.
\item<2-> It also creates $p$ fixed clusters (given as an input) and randomly assign arms into each of them such that each cluster contains equal number of arms.
\item<3-> Divide the horizon into rounds and for each round allocate $|B_{m}|n_{m}$ pulls, where $n_{m}:=\bigg\lceil\frac{2\log{(\psi T\epsilon_{m}^{2})}}{\epsilon_{m}}\bigg\rceil$.
\item<4-> Within a round pull an arm that has the maximum UCB among all arms. 
\item<5-> The arm and cluster elimination is done in same way.
\end{itemize}
\end{frame}

\begin{frame}[allowframebreaks]
\frametitle{EClusUCB Algorithm}

\begin{algorithmic}
\State {\bf Input:} Number of clusters $p$, time horizon $T$, exploration parameters $\rho_a$, $\rho_s$ and $\psi$.
\State {\bf Initialization:} Set $m:=0$, $B_{0}:=A$, $S_0 = S$, $\epsilon_{0}:=1$, $M=\big \lfloor \frac{1}{2}\log_{2} \frac{14T}{K}\big\rfloor$, $n_{0}=\bigg\lceil\frac{2\log{(\psi T\epsilon_{0}^{2})}}{\epsilon_{0}}\bigg\rceil$ and  $N_{0}=Kn_{0}$.
\State Create a partition $S_0$ of the arms at random into $p$ clusters of size up to $\ell=\bigg\lceil \frac{K}{p} \bigg\rceil$ each.
\State Pull each arm once
\For{$t=K+1,..,T$}	
\State Pull arm $i\in B_m$ such that arg$\max_{j\in B_{m}}\bigg\lbrace \hat{r}_{j} + \sqrt{\frac{\rho_{s}\log{(\psi T\epsilon_{m}^{2})}}{2 z_{j}}} \bigg\rbrace$, where $z_j$ is the number of times arm $j$ has been pulled
\State $t:=t+1$
\ArmElim
\State For each cluster $s_k \in S_{m}$, delete arm ${i}\in s_{k}$ from $B_{m}$ if
\begin{align*}
\hat{r}_{i} + \sqrt{\frac{\rho_{a}\log{(\psi T\epsilon_{m}^{2})}}{2 z_{i}}}  < \max_{{j}\in s_{k}}\bigg\lbrace\hat{r}_{j} -\sqrt{\frac{\rho_{a}\log{(\psi T\epsilon_{m}^{2})}}{2 z_{j}}} \bigg\rbrace
\end{align*}
% where $\rho_{a}=\frac{1}{w_{m}}$ and remove all such arms from $B_{m}$.
\EndArmElim
\ClusElim
\State Delete cluster $s_{k}\in S_{m}$ and remove all arms $i\in s_{k}$ from $B_{m}$ if 
\begin{align*}
 \max_{{i}\in s_{k}}\bigg\lbrace\hat{r}_{i} + \sqrt{\frac{\rho_{s}\log{(\psi T\epsilon_{m}^{2})}}{2 z_{i}}}\bigg\rbrace 
 < \max_{{j}\in B_{m}} \bigg\lbrace\hat{r}_{j} - \sqrt{\frac{\rho_{s} \log{(\psi T\epsilon_{m}^{2})}}{2 z_{j}}}\bigg\rbrace.
\end{align*}
%  and remove all such arms in the cluster $s_{k}$ from $B_{m}$ to obtain $B_{m+1}$.
\EndClusElim

\If{$t\geq N_{m}$ and $m\leq M$}
%\ResParam
\State $\epsilon_{m+1}:=\frac{\epsilon_{m}}{2}$\vspace{0.5ex}
\State $B_{m+1}:=B_{m}$
\State $n_{m+1}:=\bigg\lceil\frac{2\log{(\psi T\epsilon_{m+1}^{2})}}{\epsilon_{m+1}}\bigg\rceil$
\State $N_{m+1}:=t+|B_{m+1}| n_{m+1}$
\State $m:=m+1$
%\EndResParam
\State Stop if $|B_{m}|=1$ and pull ${i}\in B_{m}$ till $T$ is reached.
\EndIf
\EndFor
\end{algorithmic}

\end{frame}

\begin{frame}
\frametitle{UCB1, MOSS, OCUCB, UCB-Imp, EClusUCB}
\begin{table}
%\caption{Cumulative Regret of Algorithms}
\begin{center}
\begin{tabular}{|c|c|}
\toprule
Algorithm  & Upper bound on Cumulative Regret\\
\midrule
UCB1        &$\min\left \lbrace O\left(\dfrac{K\log T}{\Delta} \right), O\left( \sqrt{KT\log T}\right) \right\rbrace$ \\\midrule
MOSS        &$\min\left \lbrace O\left(\dfrac{K^2\log (T\Delta^2/K)}{\Delta} \right), O\left( \sqrt{KT}\right) \right\rbrace$ \\\midrule
UCB-Improved      &$\min\left \lbrace O\left(\dfrac{K\log (T\Delta^2)}{\Delta} \right), O\left( \sqrt{KT\log K}\right) \right\rbrace$\\\midrule
\begin{alertenv} EClusUCB    \end{alertenv}  & \begin{alertenv}$\min\left \lbrace O\left(\dfrac{K\log (T\Delta^2 /\sqrt{\log K})}{\Delta} \right), O\left( \sqrt{KT\log K}\right) \right\rbrace$ \end{alertenv}\\\midrule
OCUCB        &$\min\left \lbrace O\left(\dfrac{K\log (T/H)}{\Delta} \right), O\left( \sqrt{KT}\right) \right\rbrace$ \\\bottomrule
\end{tabular}
\end{center}
\end{table}
\end{frame}

\section{Experiments}
%\begin{frame}
%\frametitle{Finally, experiment!!!}
%\begin{figure}[tbp]
%    \centering
%    \begin{tabular}{cc}
%    \subfigure[0.32\textwidth][Expt-$1$: Arithmetic Progression (Gaussian)]
%    {
%    		\pgfplotsset{
%		tick label style={font=\Large},
%		label style={font=\Large},
%		legend style={font=\Large},
%		}
%        \begin{tikzpicture}[scale=0.5]
%      	\begin{axis}[
%		xlabel={Time-step},
%		ylabel={Error Percentage},
%		grid=major,
%        %clip mode=individual,grid,grid style={gray!30},
%        clip=true,
%        %clip mode=individual,grid,grid style={gray!30},
%  		legend style={at={(0.5,1.2)},anchor=north, legend columns=3} ]
%      	% UCB
%		\addplot table{results/budgetTestAP/APT12_comp_subsampled.txt};
%		\addplot table{results/budgetTestAP/AugUCBV1_comp_subsampled.txt};
%		%\addplot table{results/budgetTestAP/AugUCBV_1_13_comp_subsampled.txt};
%		\addplot table{results/budgetTestAP/UCBEM1_comp_subsampled.txt};
%		\addplot table{results/budgetTestAP/UCBEMV1_comp_subsampled.txt};
%		\addplot table{results/budgetTestAP/SR1_comp_subsampled.txt};
%		\addplot table{results/budgetTestAP/UA1_comp_subsampled.txt};
%		%\addplot table{results/budgetTestAP/AugUCBM12_comp_subsampled.txt};
%		%\addplot table{results/budgetTestAP/AugUCBV1_comp_subsampled.txt};
%      	%\legend{APT,AugUCB,UCBE,UCBEV,CSAR,Unif Alloc,AugUCBM,AugUCBV}
%      	\legend{APT,AugUCB,UCBE,UCBEV,CSAR,UA}
%      	\end{axis}
%      	\end{tikzpicture}
%  		\label{Fig:budgetExpt1}
%    }
%    &
%    \subfigure[0.32\textwidth][Expt-$2$: Geometric Progression (Gaussian)]
%    {
%    	\pgfplotsset{
%		tick label style={font=\Large},
%		label style={font=\Large},
%		legend style={font=\Large},
%		}
%        \begin{tikzpicture}[scale=0.5]
%        \begin{axis}[
%		xlabel={Time-step},
%		ylabel={Error Percentage},
%        %clip mode=individual,grid,grid style={gray!30},
%		grid=major,
%		clip=true,
%  		legend style={at={(0.5,1.2)},anchor=north, legend columns=3} ]
%        % UCB
%		\addplot table{results/budgetTestGP/APT12_comp_subsampled.txt};
%		\addplot table{results/budgetTestGP/AugUCBV1_comp_subsampled.txt};
%		%\addplot table{results/budgetTestGP/AugUCBV_1_13_comp_subsampled.txt};
%		\addplot table{results/budgetTestGP/UCBEM1_comp_subsampled.txt};
%		\addplot table{results/budgetTestGP/UCBEMV1_comp_subsampled.txt};
%		\addplot table{results/budgetTestGP/SR1_comp_subsampled.txt};
%		\addplot table{results/budgetTestGP/UA1_comp_subsampled.txt};
%		%\addplot table{results/budgetTestGP/AugUCBM12_comp_subsampled.txt};
%		%\addplot table{results/budgetTestGP/AugUCBV1_comp_subsampled.txt};
%        %\legend{APT,AugUCB,UCBE,UCBEV,CSAR,Unif Alloc,AugUCBM,AugUCBV}
%        \legend{APT,AugUCB,UCBE,UCBEV,CSAR,UA}
%      	\end{axis}
%      	\label{Fig:budgetExpt2}
%        \end{tikzpicture}
%    }
%    \end{tabular}
%\end{figure}
%\end{frame}
%
%\begin{frame}
%\frametitle{Finally, experiment!!!}
%\begin{figure}[tbp]
%    \centering
%    \begin{tabular}{cc}
%    \subfigure[0.32\textwidth][Expt-$3$: Three Group Setting (Gaussian)]
%    {
%    		\pgfplotsset{
%		tick label style={font=\Large},
%		label style={font=\Large},
%		legend style={font=\Large},
%		}
%        \begin{tikzpicture}[scale=0.5]
%        \begin{axis}[
%		xlabel={Time-step},
%		ylabel={Error Percentage},
%        %clip mode=individual,grid,grid style={gray!30},
%       	grid=major,
%       	clip=true,
%  		legend style={at={(0.5,1.2)},anchor=north, legend columns=3} ]
%      	% UCB
%		\addplot table{results/budgetTestGR1/APT1_comp_subsampled.txt};
%		\addplot table{results/budgetTestGR1/AugUCB1_comp_subsampled.txt};
%		\addplot table{results/budgetTestGR1/UCBEM1_comp_subsampled.txt};
%		\addplot table{results/budgetTestGR1/UCBEMV1_comp_subsampled.txt};
%		\addplot table{results/budgetTestGR1/SR1_comp_subsampled.txt};
%		\addplot table{results/budgetTestGR1/UA1_comp_subsampled.txt};
%        \legend{APT,AugUCB,UCBE,UCBEV,CSAR,UA}
%      	\end{axis}
%      	\end{tikzpicture}
%   		\label{Fig:budgetExpt3} 
%    }
%    &
%    \subfigure[0.32\textwidth][Expt-$4$: Two Group Setting (Gaussian) ]
%    {
%    	\pgfplotsset{
%		tick label style={font=\Large},
%		label style={font=\Large},
%		legend style={font=\Large},
%		}
%        \begin{tikzpicture}[scale=0.5]
%        \begin{axis}[
%		xlabel={Time-step},
%		ylabel={Error Percentage},
%        %clip mode=individual,grid,grid style={gray!30},
%		grid=major,
%		clip=true,
%  		legend style={at={(0.5,1.2)},anchor=north, legend columns=3} ]
%        % UCB
%		\addplot table{results/budgetTestGR2/APT1_comp_subsampled.txt};
%		\addplot table{results/budgetTestGR2/AugUCBV1_comp_subsampled.txt};
%		%\addplot table{results/budgetTestGP/AugUCBV_1_13_comp_subsampled.txt};
%		\addplot table{results/budgetTestGR2/UCBEM1_comp_subsampled.txt};
%		\addplot table{results/budgetTestGR2/UCBEMV1_comp_subsampled.txt};
%		\addplot table{results/budgetTestGR2/SR1_comp_subsampled.txt};
%		\addplot table{results/budgetTestGR2/UA1_comp_subsampled.txt};
%		%\addplot table{results/budgetTestGP/AugUCBM12_comp_subsampled.txt};
%		%\addplot table{results/budgetTestGP/AugUCBV1_comp_subsampled.txt};
%        %\legend{APT,AugUCB,UCBE,UCBEV,CSAR,Unif Alloc,AugUCBM,AugUCBV}
%        \legend{APT,AUgUCB,UCBE,UCBEV,CSAR,UA}
%        %\legend{APT,AugUCB,UCBE,UCBEV,CSAR,Unif Alloc}
%      	\end{axis}
%      	\label{Fig:budgetExpt4}
%        \end{tikzpicture}
%    }
%	\end{tabular}
%\end{figure}
%\end{frame}    
%    


\begin{frame}
\frametitle{Finally, experiment!!!}
\begin{itemize}
\item<1-> We compare with APT, AugUCB, UCBE, UCBEV, CSAR, UA.
\item<2-> Note that UCBE and UCBEV require access to $H_1$ and $H_{\sigma, 1}$ as input and hence not implementable in real life. 
\item<2-> By access we mean that an oracle supplies them the $H_1$ or $H_{\sigma, 1}$. They do not have access to individual means and variances.
\item<3-> APT, AugUCB, CSAR, UA do not require access to $H_1$ or $H_{\sigma, 1}$.
\item<4-> UCBE, UCBEV, CSAR and UA come from the pure exploration lineage and are modified suitably to perform in TBP setting.
\end{itemize}
\end{frame}

\begin{frame}
\frametitle{Experimental Setup}
\begin{itemize}
\item<1-> This setup involves Gaussian reward distributions with $K=100, T=10000$ and $\tau=0.5$ with the reward means set in two groups.
\item<2-> The first $10$ arms partitioned into two groups; the respective means are $r_{1:5}=0.45$, $r_{6:10}=0.55$.
\item<3-> The means of arms $i=11:100$ are chosen same as $r_{11:100}=0.4$.
\item<3-> Variances are set as $\sigma_{1:5}^{2}=0.3$ and $\sigma_{6:10}^{2}=0.8$;  $\sigma_{11:100}^{2}$ are independently and uniformly chosen in the interval $[0.2,0.3]$. 
\end{itemize}
\end{frame}


\begin{frame}
\frametitle{Experimental Result}
\begin{figure}[tbp]
    \centering
    \begin{tabular}{cc}
    \subfigure[0.32\textwidth][Expt-$1$: Two Group Setting (Advance) ]
    {
    	\pgfplotsset{
		tick label style={font=\Large},
		label style={font=\Large},
		legend style={font=\Large},
		}
        \begin{tikzpicture}[scale=0.5]
        \begin{axis}[
		xlabel={Time-step},
		ylabel={Error Percentage},
        %clip mode=individual,grid,grid style={gray!30},
		grid=major,
		clip=true,
  		legend style={at={(0.5,1.2)},anchor=north, legend columns=3} ]
        % UCB
		\addplot table{results/budgetTestGR4/APT1_comp_subsampled.txt};
		\addplot table{results/budgetTestGR4/AugUCB1_comp_subsampled.txt};
		\addplot table{results/budgetTestGR4/UCBEM1_comp_subsampled.txt};
		\addplot table{results/budgetTestGR4/UCBEMV1_comp_subsampled.txt};
		\addplot table{results/budgetTestGR4/SR1_comp_subsampled.txt};
		\addplot table{results/budgetTestGR4/UA1_comp_subsampled.txt};
        \legend{APT,AugUCB,UCBE,UCBEV,CSAR,UA}
      	\end{axis}
      	\label{Fig:budgetExpt5}
        \end{tikzpicture}
    }
    &
    \subfigure[0.32\textwidth][Expt-$2$: Two Group Setting (Advance) ]
    {
    	\pgfplotsset{
		tick label style={font=\Large},
		label style={font=\Large},
		legend style={font=\Large},
		}
        \begin{tikzpicture}[scale=0.5]
        \begin{axis}[
		xlabel={Time-step},
		ylabel={Error Percentage},
        %clip mode=individual,grid,grid style={gray!30},
		grid=major,
		clip=true,
  		legend style={at={(0.5,1.2)},anchor=north, legend columns=2} ]
        % UCB
		\addplot table{results/budgetTestGR3/testUCBEMV1_0.25_comp_subsampled.txt};
		\addplot table{results/budgetTestGR4/AugUCB1_comp_subsampled.txt};
		%\addplot table{results/budgetTestGP/AugUCBV_1_13_comp_subsampled.txt};
		%\addplot table{results/budgetTestGR3/testUCBEMV1_0.25_comp_subsampled.txt};
		\addplot table{results/budgetTestGR3/testUCBEMV1_256_comp_subsampled.txt};
		\addplot table{results/budgetTestGR4/UCBEMV1_comp_subsampled.txt};
		%\addplot table{results/budgetTestGR3/testUCBEMV1_64_comp_subsampled.txt};
		%\addplot table{results/budgetTestGP/AugUCBM12_comp_subsampled.txt};
		%\addplot table{results/budgetTestGP/AugUCBV1_comp_subsampled.txt};
        %\legend{APT,AugUCB,UCBE,UCBEV,CSAR,Unif Alloc,AugUCBM,AugUCBV}
        \legend{UCBEV($0.25$), AugUCB, UCBEV($256$), UCBEV($1$)}
        %\legend{UCBEV($0.06$),AUgUCB,UCBEV($0.25$),UCBEV($1$),UCBEV($64$),UCBEV($256$)}
        %\legend{APT,AugUCB,UCBE,UCBEV,CSAR,Unif Alloc}
      	\end{axis}
      	\label{Fig:budgetExpt6}
        \end{tikzpicture}
    }
    \end{tabular}
\end{figure}
\end{frame}

\section{References}
\begin{frame}[allowframebreaks]
\frametitle{References}
\bibliographystyle{plainnat} 
\bibliography{biblio}
\end{frame}


%------------------------------------------------

\begin{frame}
\Huge{\centerline{Thank You}}
\end{frame}

%----------------------------------------------------------------------------------------

\end{document} 