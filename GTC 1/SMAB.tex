\begin{frame}
\frametitle{Stochastic Multi-Armed Bandit Problem}
\begin{itemize}
\item<1-> In stochastic multi-armed bandit problem, we are presented with a finite set of actions or arms. 
\item<2-> The rewards for each of the arms are identical and independent random variables drawn from distribution specific to the arm.
\item<3-> The learner does not know the mean of the distributions, denoted by $r_{i},\forall i\in A$. 
\item<4-> The learner has to find the optimal arm the mean of whose distribution is denoted by $r^{*}$ such that $r^{*}> r_{i}, \forall i\in A$.
\item<5-> The distributions for each of the arms are fixed throughout the time horizon. 
\end{itemize}
\end{frame}

\begin{frame}
\frametitle{Basic Notations}
\begin{itemize}
\item<1-> Goal: To minimize Regret
\item<2-> Average reward of best action is $r^{*}$ and any other action $i$ as $r_{i}$. There are $K$ total actions. $T_{i}(n)$ is number of times tried action $i$ is executed till $n$-timesteps.
\item<3-> Cumulative Regret: The loss we suffer because of not pulling the optimal arm till the total number of timesteps  $T$. 
\begin{align*}
R_{T}=r^{*}T - \sum_{i\in A} r_{i}T_{i}(T),
\end{align*}
\item<4-> The expected regret of an algorithm after $T$ rounds can be written as
\begin{align*}
\E[R_{T}]= \sum_{i=1}^K \E[T_{i}(T)] \Delta_i,
\end{align*}
\item<4-> $\Delta_{i}=r^{*}-r_{i}$ denotes the gap between the means of the optimal arm and of the $i$-th arm. 
\end{itemize}
\end{frame}

\begin{frame}
\frametitle{Another Notion of Regret}
\begin{itemize}
\item<1-> Goal: To minimize Regret
\item<2-> Can we have a policy which achieves the minimum regret among all the possible environments available?
\item<3-> This is called the worst case gap-independent regret.
\item<4-> It is generally found by setting all the gaps to equal values of $\min_{i\in A}\Delta_{i}=\Delta=\sqrt{\dfrac{f(K)}{T}}$.
\item<5-> Also we will define the hardness parameter $H$ as $H=\sum_{i=1}^{K}\dfrac{1}{\Delta_{i}^2}$
\end{itemize}
\end{frame}