\begin{frame}
\frametitle{Literature Survey}
\begin{itemize}
\item<1-> A considerable amount of research has been conducted on SMAB. We can divide the gamut of literature on SMAB into broadly two categories:
\begin{itemize}
\item<2-> \textbf{Frequentist approach:} In this approach for each of the arms compute a dynamic allocation index that depends only on the number of draws on the arm and exploration parameters and choose the arm with the maximal index. Eg:  \emph{UCB} variants
\item<3-> \textbf{Bayesian Approach:} In this approach we start with a prior guess over the performance of each of the arms. Then we pull an arm by behaving greedily based on our guess, receive the reward and then we update our prior for the arm. Eg: TS 
\end{itemize} 
\item<4-> We will be focusing on UCB based approaches in our work.
\end{itemize}
\end{frame}

\begin{frame}
\frametitle{Literature Survey}
\begin{itemize}
\item<1-> The literature under UCB can be broadly classified into two categories:
\begin{itemize}
\item<2-> \textbf{Mean-based estimation: } In this approach at every timestep we choose an arm based on $\hat{r}_i$ and its confidence interval $c_i$. Eg: UCB1 \cite{auer2002finite}, MOSS \cite{audibert2009minimax}, UCB-Improved \cite{auer2010ucb}
\item<3-> \textbf{Mean and Variance based Estimation: } Here, at every timestep we choose an arm based on $\hat{r}_i$, $\hat{V}_i$ and its confidence interval $c_i$. Eg: UCB-Normal \cite{auer2002finite}l, UCB-V \cite{audibert2009exploration}.
%\item<4-> \textbf{Divergence based methods: } In this approach at every timestep we choose an arm that minimizes the divergence function. Eg: KL-UCB \cite{garivier2011kl}, DMED \cite{honda2010asymptotically}. 
\end{itemize}
\item<4-> We will be focusing on Mean based estimation.
\end{itemize} 
\end{frame}

\begin{frame}
\frametitle{UCB1 Algorithm (\cite{auer2002finite})}
\begin{algorithm}[H]
\caption{UCB1}
\begin{algorithmic}[1]
\State Pull each arm once
 \For{$t=K+1,..., T$}
\State Pull the arm such that $\max_{i\in A}\bigg\lbrace\hat{r}_{i} + \sqrt{\dfrac{2\log t}{s_i}}\bigg\rbrace$
\State $t:=t+1 $
 \EndFor
\end{algorithmic}
\end{algorithm}

\begin{itemize}
\item<1-> Maintain an upper confidence bound ($c_i$) for each of the arms
\item<1-> This $c_i$ will help in sufficiently exploring sub-optimal arms and then exploiting the optimal arm.
\item<1-> The gap-independent regret bound of $O\left( \sqrt{KT\log T}\right) $ and gap-dependent bound of $O\left( \dfrac{K \log (T)}{\Delta} \right)$.
\end{itemize}
\end{frame}

\begin{frame}
\frametitle{Minimax Optimal Strategy in the Stochastic Case (\cite{audibert2009minimax})}
\begin{algorithm}[H]
\caption{MOSS}
\begin{algorithmic}[1]
\State Pull each arm once
 \For{$t=K+1,..., T$}
\State Pull the arm such that $\max_{i\in A}\bigg\lbrace\hat{r}_{i} + \sqrt{\dfrac{\max\lbrace 0,\log(\frac{T}{K s_i})\rbrace}{s_i}}\bigg\rbrace$
\State $t:=t+1 $
 \EndFor
\end{algorithmic}
\end{algorithm}
\begin{itemize}
\item<1-> UCB1 suffers from a worst case regret of $O\left( \sqrt{KT\log T }\right) $.
\item<1-> MOSS corrects this and gives us a gap-independent regret bound of $O\left( \sqrt{KT}\right)$ and gap-dependent bound of $O\left( \dfrac{K^2 \log (\frac{T\Delta^2}{K})}{\Delta}\right)$.
%UCB1 performs badly when the gaps are closer to $\sqrt{\dfrac{K}{T}}$.
\end{itemize}
\end{frame}

\begin{frame}
\frametitle{Approach of UCB-Improved}
\begin{itemize}
\item<1-> The basic idea of UCB-Improved is to divide the horizon into phases or rounds and initialize parameters.
\item<2-> Pull all surviving arms equal number of times during a round.
\item<3-> At the end of the round eliminate some arms based on some criteria.
\item<4-> Reset parameters and proceed to next round.
\item<5-> UCB-Imp achieves a gap-independent regret bound of $O\left( \sqrt{KT\log K}\right)$ and gap-dependent bound of $O\left( \dfrac{K \log (T\Delta^2)}{\Delta}\right)$.
\end{itemize}
\end{frame}

\begin{frame}
\frametitle{UCB-Improved (\cite{auer2010ucb})}
\begin{algorithm}[H]
\caption{UCB-Improved}
\begin{algorithmic}[1]
\State {\bf Input:} Time horizon $T$
\State {\bf Initialization:} Set $B_{0}:=A$ and $\tilde{\Delta}_{0}:=1$.
\For{$m=0,1,..\big \lfloor \dfrac{1}{2}\log_{2} \dfrac{T}{e}\big\rfloor$}	
\State Pull each arm in $B_m$, $n_{m}=\bigg\lceil\dfrac{2\log{( T\tilde{\Delta}_{m}^{2})}}{\tilde{\Delta}_{m}}\bigg\rceil$ number of times.
%so that the total  it has been pulled is
\ArmElim
\State For each $i \in B_{m}$, delete arm ${i}$ from $B_{m}$ if,
\begin{align*}
\bar{X}_{i} + \sqrt{\dfrac{\log{(T\tilde{\Delta}_{m}^{2})}}{2 n_{m}}}  < \max_{{j}\in B_{m}}\bigg\lbrace\bar{X}_{j} -\sqrt{\dfrac{\log{( T\tilde{\Delta}_{m}^{2})}}{2 n_{m}}} \bigg\rbrace
\end{align*}
\EndArmElim
%\ResParam
\State Set $\tilde{\Delta}_{m+1}:=\dfrac{\tilde{\Delta}_{m}}{2}$, Set $B_{m+1}:=B_{m}$
%\EndResParam
\State Stop if $|B_{m}|=1$ and pull ${i}\in B_{m}$ till $n$ is reached.
\EndFor
\end{algorithmic}
\end{algorithm}
\end{frame}

\begin{frame}
\frametitle{Some technical details of UCB-Improved}
\begin{itemize}
\item<1-> We do not know the true means $r_i ,\forall i\in A$ of the distributions so we estimate it by the $\tilde{\Delta}$ by initializing it from $1$.
\item<2-> All rewards are assume to be bounded between $[0,1]$ and so $\Delta_{i}\in [0,1],\forall i\in A$ as well.
\item<3-> As opposed to UCB1, MOSS and OCUCB, UCB-Improved has fixed confidence interval  $c_{m}=\sqrt{\dfrac{\log{(T\tilde{\Delta}_{m}^{2})}}{2 n_{m}}}$ for all arms in a particular phase.
\item<4-> $c_m$ ensures that whenever $\tilde{\Delta}_{m}<\dfrac{\Delta_i}{2}$ in the $m$-th round, the arm $i$ gets eliminated.
\end{itemize}
\end{frame}

\begin{frame}
\frametitle{Optimally confident UCB (\cite{lattimore2015optimally})}
\begin{algorithm}[H]
\caption{MOSS}
\begin{algorithmic}[1]
\State \textbf{Input: } K,T, $\alpha$, $\psi$
\State Pull each arm once
 \For{$t=K+1,..., T$}
\State Pull the arm such that $\max_{i\in A}\bigg\lbrace\hat{r}_{i} + \sqrt{\alpha\dfrac{\max\lbrace 0,\log(\frac{\psi T}{ s_i})\rbrace}{s_i}}\bigg\rbrace$
\State $t:=t+1 $
 \EndFor
\end{algorithmic}
\end{algorithm}
\begin{itemize}
\item<1-> UCB1 is too conservative in exploiting, MOSS is not conservative enough and tends to explore more often than required. 
\item<1-> OCUCB correctly balances this and achieves a gap-independent regret bound  $O\left(\sqrt{KT}\right)$ and gap-dependent bound $O \left(\dfrac{K\log (T/H)}{\Delta}\right)$.
\end{itemize}
\end{frame}

\begin{frame}
\frametitle{Comparison of UCB1, MOSS, OCUCB, UCB-Improved}
\begin{table}
\caption{Cumulative Regret of Algorithms}
\begin{center}
\begin{tabular}{|c|c|}
\toprule
Algorithm  & Upper bound on Cumulative Regret\\
\midrule
UCB1        &$\min\left \lbrace O\left(\dfrac{K\log T}{\Delta} \right), O\left( \sqrt{KT\log T}\right) \right\rbrace$ \\\midrule
MOSS        &$\min\left \lbrace O\left(\dfrac{K^2\log (T\Delta^2/K)}{\Delta} \right), O\left( \sqrt{KT}\right) \right\rbrace$ \\\midrule
UCB-Improved      &$\min\left \lbrace O\left(\dfrac{K\log (T\Delta^2)}{\Delta} \right), O\left( \sqrt{KT\log K}\right) \right\rbrace$\\\midrule
OCUCB        &$\min\left \lbrace O\left(\dfrac{K\log (T/H)}{\Delta} \right), O\left( \sqrt{KT}\right) \right\rbrace$ \\\bottomrule
\end{tabular}
\end{center}
\end{table}
\end{frame}

